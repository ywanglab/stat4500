% Options for packages loaded elsewhere
\PassOptionsToPackage{unicode}{hyperref}
\PassOptionsToPackage{hyphens}{url}
\PassOptionsToPackage{dvipsnames,svgnames,x11names}{xcolor}
%
\documentclass[
  letterpaper,
  DIV=11,
  numbers=noendperiod]{scrreprt}

\usepackage{amsmath,amssymb}
\usepackage{iftex}
\ifPDFTeX
  \usepackage[T1]{fontenc}
  \usepackage[utf8]{inputenc}
  \usepackage{textcomp} % provide euro and other symbols
\else % if luatex or xetex
  \usepackage{unicode-math}
  \defaultfontfeatures{Scale=MatchLowercase}
  \defaultfontfeatures[\rmfamily]{Ligatures=TeX,Scale=1}
\fi
\usepackage{lmodern}
\ifPDFTeX\else  
    % xetex/luatex font selection
\fi
% Use upquote if available, for straight quotes in verbatim environments
\IfFileExists{upquote.sty}{\usepackage{upquote}}{}
\IfFileExists{microtype.sty}{% use microtype if available
  \usepackage[]{microtype}
  \UseMicrotypeSet[protrusion]{basicmath} % disable protrusion for tt fonts
}{}
\makeatletter
\@ifundefined{KOMAClassName}{% if non-KOMA class
  \IfFileExists{parskip.sty}{%
    \usepackage{parskip}
  }{% else
    \setlength{\parindent}{0pt}
    \setlength{\parskip}{6pt plus 2pt minus 1pt}}
}{% if KOMA class
  \KOMAoptions{parskip=half}}
\makeatother
\usepackage{xcolor}
\setlength{\emergencystretch}{3em} % prevent overfull lines
\setcounter{secnumdepth}{5}
% Make \paragraph and \subparagraph free-standing
\ifx\paragraph\undefined\else
  \let\oldparagraph\paragraph
  \renewcommand{\paragraph}[1]{\oldparagraph{#1}\mbox{}}
\fi
\ifx\subparagraph\undefined\else
  \let\oldsubparagraph\subparagraph
  \renewcommand{\subparagraph}[1]{\oldsubparagraph{#1}\mbox{}}
\fi


\providecommand{\tightlist}{%
  \setlength{\itemsep}{0pt}\setlength{\parskip}{0pt}}\usepackage{longtable,booktabs,array}
\usepackage{calc} % for calculating minipage widths
% Correct order of tables after \paragraph or \subparagraph
\usepackage{etoolbox}
\makeatletter
\patchcmd\longtable{\par}{\if@noskipsec\mbox{}\fi\par}{}{}
\makeatother
% Allow footnotes in longtable head/foot
\IfFileExists{footnotehyper.sty}{\usepackage{footnotehyper}}{\usepackage{footnote}}
\makesavenoteenv{longtable}
\usepackage{graphicx}
\makeatletter
\def\maxwidth{\ifdim\Gin@nat@width>\linewidth\linewidth\else\Gin@nat@width\fi}
\def\maxheight{\ifdim\Gin@nat@height>\textheight\textheight\else\Gin@nat@height\fi}
\makeatother
% Scale images if necessary, so that they will not overflow the page
% margins by default, and it is still possible to overwrite the defaults
% using explicit options in \includegraphics[width, height, ...]{}
\setkeys{Gin}{width=\maxwidth,height=\maxheight,keepaspectratio}
% Set default figure placement to htbp
\makeatletter
\def\fps@figure{htbp}
\makeatother
% definitions for citeproc citations
\NewDocumentCommand\citeproctext{}{}
\NewDocumentCommand\citeproc{mm}{%
  \begingroup\def\citeproctext{#2}\cite{#1}\endgroup}
\makeatletter
 % allow citations to break across lines
 \let\@cite@ofmt\@firstofone
 % avoid brackets around text for \cite:
 \def\@biblabel#1{}
 \def\@cite#1#2{{#1\if@tempswa , #2\fi}}
\makeatother
\newlength{\cslhangindent}
\setlength{\cslhangindent}{1.5em}
\newlength{\csllabelwidth}
\setlength{\csllabelwidth}{3em}
\newenvironment{CSLReferences}[2] % #1 hanging-indent, #2 entry-spacing
 {\begin{list}{}{%
  \setlength{\itemindent}{0pt}
  \setlength{\leftmargin}{0pt}
  \setlength{\parsep}{0pt}
  % turn on hanging indent if param 1 is 1
  \ifodd #1
   \setlength{\leftmargin}{\cslhangindent}
   \setlength{\itemindent}{-1\cslhangindent}
  \fi
  % set entry spacing
  \setlength{\itemsep}{#2\baselineskip}}}
 {\end{list}}
\usepackage{calc}
\newcommand{\CSLBlock}[1]{\hfill\break#1\hfill\break}
\newcommand{\CSLLeftMargin}[1]{\parbox[t]{\csllabelwidth}{\strut#1\strut}}
\newcommand{\CSLRightInline}[1]{\parbox[t]{\linewidth - \csllabelwidth}{\strut#1\strut}}
\newcommand{\CSLIndent}[1]{\hspace{\cslhangindent}#1}

\KOMAoption{captions}{tableheading}
\makeatletter
\@ifpackageloaded{tcolorbox}{}{\usepackage[skins,breakable]{tcolorbox}}
\@ifpackageloaded{fontawesome5}{}{\usepackage{fontawesome5}}
\definecolor{quarto-callout-color}{HTML}{909090}
\definecolor{quarto-callout-note-color}{HTML}{0758E5}
\definecolor{quarto-callout-important-color}{HTML}{CC1914}
\definecolor{quarto-callout-warning-color}{HTML}{EB9113}
\definecolor{quarto-callout-tip-color}{HTML}{00A047}
\definecolor{quarto-callout-caution-color}{HTML}{FC5300}
\definecolor{quarto-callout-color-frame}{HTML}{acacac}
\definecolor{quarto-callout-note-color-frame}{HTML}{4582ec}
\definecolor{quarto-callout-important-color-frame}{HTML}{d9534f}
\definecolor{quarto-callout-warning-color-frame}{HTML}{f0ad4e}
\definecolor{quarto-callout-tip-color-frame}{HTML}{02b875}
\definecolor{quarto-callout-caution-color-frame}{HTML}{fd7e14}
\makeatother
\makeatletter
\@ifpackageloaded{bookmark}{}{\usepackage{bookmark}}
\makeatother
\makeatletter
\@ifpackageloaded{caption}{}{\usepackage{caption}}
\AtBeginDocument{%
\ifdefined\contentsname
  \renewcommand*\contentsname{Table of contents}
\else
  \newcommand\contentsname{Table of contents}
\fi
\ifdefined\listfigurename
  \renewcommand*\listfigurename{List of Figures}
\else
  \newcommand\listfigurename{List of Figures}
\fi
\ifdefined\listtablename
  \renewcommand*\listtablename{List of Tables}
\else
  \newcommand\listtablename{List of Tables}
\fi
\ifdefined\figurename
  \renewcommand*\figurename{Figure}
\else
  \newcommand\figurename{Figure}
\fi
\ifdefined\tablename
  \renewcommand*\tablename{Table}
\else
  \newcommand\tablename{Table}
\fi
}
\@ifpackageloaded{float}{}{\usepackage{float}}
\floatstyle{ruled}
\@ifundefined{c@chapter}{\newfloat{codelisting}{h}{lop}}{\newfloat{codelisting}{h}{lop}[chapter]}
\floatname{codelisting}{Listing}
\newcommand*\listoflistings{\listof{codelisting}{List of Listings}}
\makeatother
\makeatletter
\makeatother
\makeatletter
\@ifpackageloaded{caption}{}{\usepackage{caption}}
\@ifpackageloaded{subcaption}{}{\usepackage{subcaption}}
\makeatother
\ifLuaTeX
  \usepackage{selnolig}  % disable illegal ligatures
\fi
\IfFileExists{bookmark.sty}{\usepackage{bookmark}}{\usepackage{hyperref}}
\IfFileExists{xurl.sty}{\usepackage{xurl}}{} % add URL line breaks if available
\urlstyle{same} % disable monospaced font for URLs
\hypersetup{
  pdftitle={stat4500notes},
  pdfauthor={Yi Wang},
  colorlinks=true,
  linkcolor={blue},
  filecolor={Maroon},
  citecolor={Blue},
  urlcolor={Blue},
  pdfcreator={LaTeX via pandoc}}

\title{stat4500notes}
\author{Yi Wang}
\date{2023-09-19}

\begin{document}
\maketitle
\renewcommand*\contentsname{Table of contents}
{
\hypersetup{linkcolor=}
\setcounter{tocdepth}{2}
\tableofcontents
}
\bookmarksetup{startatroot}

\chapter*{Preface}\label{preface}
\addcontentsline{toc}{chapter}{Preface}

\markboth{Preface}{Preface}

This is a Lecture note written for the course STAT 4500: Machine
Learning offered at Auburn University at Montgomery. The course uses the
textbook James et al. (2023).

This is a book wrtieen by Quarto book.

To learn more about Quarto books visit
\url{https://quarto.org/docs/books}.

\bookmarksetup{startatroot}

\chapter{Setting up Python Computing
Environment}\label{setting-up-python-computing-environment}

\section{on Your own computer}\label{on-your-own-computer}

\begin{enumerate}
\def\labelenumi{\arabic{enumi}.}
\item
  you can either \texttt{git\ clone} or download a zipped file
  containing the codes from the site:
  \url{https://github.com/intro-stat-learning/ISLP_labs/tree/stable}. If
  downloaded a zipped file of the codes, unzipped the file to a folder,
  for example, named \texttt{islp}. If \texttt{git\ clone} (preferred,
  you need to have Git installed on your computer, check this link for
  how to install \texttt{Git}
  \url{https://ywanglab.github.io/stat1010/git.html}), do
  \texttt{git\ clone\ https://github.com/intro-stat-learning/ISLP\_labs.git}
\item
  Download and install the following software:

  \begin{itemize}
  \item
    \textbf{Anaconda}: Download anaconda and install using default
    installation options
  \item
    \textbf{Visual Studio Code} (VSC): Download VSC and install
  \item
    start VSC and install VSC extensions in VSC: Python, Jupyter,
    intellicode
  \item
    (optional) \textbf{Quarto} for authoring: Download Quarto and
    install
  \end{itemize}
\item
  Create a virtual environment named \texttt{islp} for Python. Start an
  anaconda terminal.

\begin{verbatim}
  conda create -n islp python==3.10
  conda activate islp
  conda install pip ipykernel
  pip install -r https://raw.githubusercontent.com/intro-stat-learning/ISLP_labs/v2.1.2/requirements.txt
\end{verbatim}
\item
  You are ready to run the codes using \texttt{VSC} or
  \texttt{jupyter\ lab}.

  \begin{itemize}
  \item
    Activate the venv: \texttt{conda\ activate\ islp}
  \item
    Start a Anaconda terminal, navigate to the folder using the command
    \texttt{cd\ path/to/islp}, where \texttt{path/to/islp} means the
    file path to the folder \texttt{islp}, such as
    \texttt{\textbackslash{}Users\textbackslash{}ywang2\textbackslash{}islp}.
    Start VSC by typing \texttt{code\ .} in the anaconda terminal.
  \item
    open/create a \texttt{.ipynb} or \texttt{.py} file.
  \item
    Select the kernel \texttt{islp}
  \item
    Run a code cell by pressing \texttt{Shift+Enter} or click the
    triangular play button.
  \item
    Continue to run other cells.
  \item
    After finishing using VSC, close the VSC, and deactivate the virtual
    environment in a conda terminal: \texttt{conda\ deactivate}
  \end{itemize}
\end{enumerate}

\section{Use Google Colab}\label{use-google-colab}

All you need is a Google account. Sign in your Google account in a
browser, and navigate to Google Colab. Google Colab supports both
\texttt{Python} and \texttt{R}. \texttt{Python} is the default engine.
Change the engine to \texttt{R} in
\texttt{Connect}-\textgreater{}\texttt{change\ runtime\ type}. Then you
are all set. Your file will be saved to your Google Drive or you can
choose to send it to your \texttt{GitHub} account (recommended).

\subsection{How to run a project file from your Google
Drive?}\label{how-to-run-a-project-file-from-your-google-drive}

Many times, when you run a python file in Colab, it needs to access
other files, such as data files in a subdirectory. In this case, it
would be convenient to have the same file structure in the Google Colab
user home directory. To do this, you can use Google Drive to store your
project folder, and then mount the Google Drive in Colab.

Let's assume the project folder name, \texttt{islp/}.Here are the steps:

\begin{enumerate}
\def\labelenumi{\arabic{enumi}.}
\item
  \texttt{git\ clone} the project folder (example:
  \texttt{git\ clone\ https://github.com/intro-stat-learning/ISLP\_labs.git})
  to your local folder. This step is only needed when you want to clone
  some remote repo from GitHub.
\item
  \textbf{Upload} the folder (ex: \texttt{islp}) to Google Drive.
\item
  \textbf{Open the file using Colab}. In Google Drive, double click on
  the \texttt{ipynb} file, example, \texttt{ch06.ipynb} (or click on the
  three dots on the right end, and choose \texttt{open\ with}, then
  \texttt{Google\ Colaborotary}), the file will be opened by Google
  Colab.
\item
  \textbf{Mount the Google Drive}. In Google Colab, with the specific
  file (example, \texttt{ch06.ipynb}) being opened, move your cursor to
  the first code cell, and then click on the folder icon (this should be
  the fourth icon) on the upper left border in the Colab browser. This
  will open the file explorer pane. Typically you would see a folder
  named \texttt{sample\_data} shown. On the top of the pane, click on
  the Google Drive icon to mount the Google Drive. Google Colab will
  insert the following code below the cursor in your opened
  \texttt{ipynb} file:

\begin{verbatim}
from google.colab import drive
drive.mount('/content/drive')
\end{verbatim}

  Run this code cell by pressing \texttt{SHIFT+ENTER}, and follow the
  prompts to complete the authentication. Wait for \textasciitilde10
  seconds, your Google Drive will be mounted in Colab, and it will be
  displayed as a folder named \texttt{drive} in the file explorer pane.
  You might need to click on the \texttt{Refresh} folder icon to see the
  folder \texttt{drive}.
\item
  Open a new code cell below the above code cell, and type the code

\begin{verbatim}
  %cd /content/drive/MyDrive/islp/
\end{verbatim}

  This is to change the directory to the project directory on the Google
  Drive. Run this code cell, and you are ready to run the file
  \texttt{ch06.ipynb} from the folder \texttt{islp} on your personal
  Google Drive, just like it's on your local computer.
\end{enumerate}

\bookmarksetup{startatroot}

\chapter{Chapter 2: Statistical
Learning}\label{chapter-2-statistical-learning}

\section{What is statistical
learning?}\label{what-is-statistical-learning}

For the input variable \(X\in \mathbb{R}^p\) and response variable
\(Y\in \mathbb{R}\), assume that \[Y=f(X) + \epsilon, \] where
\(\epsilon\) is a random variable representing \textbf{irreducible
error}. We assume \(\epsilon\) is \emph{independent} of \(X\) and
\(E[\epsilon]=0\). \(\epsilon\) may include \emph{unmeasured variables}
or \emph{unmeasurable variation}.

Statistical learning is to estimate \(f\) using various methods. Denote
the estimate by \(\hat{f}\).

\begin{itemize}
\tightlist
\item
  regression problem: when \(Y\) is a continuous (quantitative) variable
  . In this case \(f(x)=E(Y|X=x)\) is the population \emph{regression}
  function, that is, regression finds a conditional expectation of
  \(Y\).
\item
  classification problem: when \(Y\) only takes small number of discrete
  values, i.e., qualitative (categorical).
\end{itemize}

\textbf{Logistic regression} is a classification problem, but since it
estimates class probability, it may be considered as a regression
problem.

\begin{itemize}
\tightlist
\item
  supervised learning: training data
  \(\mathcal{Tr}=\{(x_i, y_i):i\in \mathbb{Z}_n\}\): linear regression,
  logistic regression
\item
  unsupervised learning: when only \(x_i\) are available. clustering
  analysis, PCA
\item
  semi-supervised learning: some data with labels (\(y_i\)), some do
  not.
\item
  reinforcement learning: learn a state-action policy function for an
  agent to interacting with an environment to maximize a reward
  function.
\end{itemize}

\section{\texorpdfstring{Why estimate
\(f\)?}{Why estimate f?}}\label{why-estimate-f}

We can use estimated \(\hat{f}\) to

\begin{itemize}
\tightlist
\item
  make predictions for a new \(X\), \[\hat{Y} =\hat{f}(X). \] The
  prediction error may be quantified as
  \[E[(Y-\hat{Y})^2] = (f(X)-\hat{f})^2 +\text{Var}[\epsilon].\] The
  first term of the error is \emph{reducible} by trying to improve
  \(\hat{f}\), where we assume \(f\), \(\hat{f}\) and \(X\) are fixed.
\item
  make inference, such as

  \begin{itemize}
  \tightlist
  \item
    Which predictors are associated with the response?
  \item
    what is the relationship between the response and each predictor?
  \item
    is the assumed relationship adequate? (linear or more complicated?)
  \end{itemize}
\end{itemize}

\section{\texorpdfstring{How to estimate
\(f\)}{How to estimate f}}\label{how-to-estimate-f}

We use obtained observations called \textbf{training data}
\(\{(x_k, y_k): k \in \mathbb{Z}_n \}\) to train an algorithm to obtain
the estimate \(\hat{f}\).

\begin{itemize}
\item
  Parametric methods: first assume there is a function form (shape) with
  some parameters. For example, a linear regression model with two
  parameters. Then use the \emph{training data} to \textbf{train} or
  \textbf{fit} the model to determine the values of the parameters.

  \textbf{Advantages}: simplify the problem of fit an arbitrary function
  to estimate a set of parameters.

  \textbf{Disadvantages}: may not be flexible unless with large number
  of parameters and/or complex function shapes.

  Example: linear regression,
\item
  Non-parametric methods: Do not explicitly assume a function form of
  \(f\). They seek to estimate \(f\) directly using data points, can be
  quite flexible and accurate.

  **Disadvantage: need large number of data points

  Example: KNN (but breakdown for higher dimention. Typically only for
  \(p\le 4\)), spline fit.
\end{itemize}

\section{How to assess model
accuracy}\label{how-to-assess-model-accuracy}

For regression problems, the most commonly used measure is the
\emph{mean squared error} (MSE), given by \[
MSE = \frac{1}{n}\sum_{i=1}^n (y_i-\hat{f}(x_i))^2
\] For classification problems, typically the following \textbf{error
rate} (classifications error) is calculated: \[
\frac{1}{n} \sum_{i=1}^{n} I(y_i\ne \hat{y}_i)
\] The accuracy on a training set can be arbitrarily increased by
increasing the model flexibility. However, we are in general interested
in the error on the test set rather on the training set, the model
accuracy should be assessed on a test set.

Flexible models tend to overfit the data, which essentially means they
follow the error or \emph{noise} too closely in the training set,
therefore cannot be generalized to \emph{unseen cases} (test set).

\section{Model Selection:}\label{model-selection}

\textbf{No free lunch theorem}

There is no single best method for all data sets, which means some
method works better than other methods for a particular dataset.
Therefore, one needs to perform model selections. Here are some
principles.

\subsection{Trade-off between Model flexibility and Model
Interpretability}\label{trade-off-between-model-flexibility-and-model-interpretability}

More flexible models have higher \emph{degree of freedom} and are less
interpretable because it's difficult to interpret the relationship
between a predictor and the response.

LASSO is less flexible than linear regression. GAM (generalized additive
model) allows some non-linearity. Full non-linear models have higher
flexibility, such as \emph{bagging, boosting, SVM}, etc.

When \emph{inference} is the goal, then there are advantages to using
simple and less flexible models for interpretability.

When \emph{prediction} is the main goal, more flexible model may be a
choice. But sometimes, we obtain more accurate prediction using a
simpler model because the underlying dataset has a simpler structure.
Therefore, it is not necessarily true that a more flexible model has a
higher prediction accuracy.

\textbf{Occam's Razor}: Among competing hypotheses that perform equally
well, the one with the fewest assumptions should be selected.

\subsection{Model Selection: the Bias-Variance
Trade-off}\label{model-selection-the-bias-variance-trade-off}

As the model flexibility increases, the training MSE (or error rate for
classificiton) will decrease, but the test MSE (error rate) in general
will not and will show a characteristic \textbf{U-shape}. This is
because when evaluated at a test point \(x_0\), the expected test MSE
can be decomposed into \[
E\left[ (y_0-\hat{f}(x_0))^2 \right] = \text{Var}[\hat{f}(x_0)] + (\text{Bias}(\hat{f}(x_0)))^2+\text{Var}[\epsilon]
\] where the expectation is over different \(\hat{f}\) on a different
training set or on a different training step if the training process is
stochastic, and \[
\text{Bias}(\hat{f}(x_0))= E[\hat{f}(x_0)]-f(x_0)
\] To obtain the least test MSE, one must trade off between variance and
bias. Less flexible model tendes to have higher bias, and more flexible
models tend to have higher variance. An optimal flexibility for the
least test MSE varies with different data sets. Non-linear data tends to
require higher optimal flexibility.

\section{Bayes Classifier}\label{bayes-classifier}

It can be shown that Bayes Classifier minimizes the classification test
error \[
\text{Ave}(I(y_0\ne \hat{y}_0)).
\] A Bayes Classifier assigns a test observation with predictor \(x_0\)
to the class for which \[
\text{Pr}(Y=j|X=x_0)
\] is largest. It's error rate is given by \[
1-E[\max_{j} \text{Pr}(Y=j|X)]
\]

where the expectation is over \(X\). The Bayes error is analogous to the
irreducible error \(\epsilon\).

Bayes Classifier is not attainable as we do not know \(\text{Pr}(Y|X)\).
We only can estimate \(\text{Pr}(Y|X)\). One way to do this is by KNN.
KNN estimate the conditional probability simply with a majority vote.
The flexibility of KNN increases as \(1/K\) increases with \(K=1\) being
the most flexible KNN. The training error is 0 for \(K=1\). A suitable
\(K\) should be chosen for an appropriate trade off between bias and
variance. The KNN classifier will classify the test point \(x_0\) based
on the probability calculated from the \(k\) nearest points. KNN
regression on the other hand will assign the test point \(x_0\) the
average value of the \(k\) nearest neighbors.

\section{Homework (* indicates
optional):}\label{homework-indicates-optional}

\begin{itemize}
\tightlist
\item
  Conceptual: 1,2,3,4*,5,6,7
\item
  Applied: 8, 9*, 10*
\end{itemize}

\section{Code Gist}\label{code-gist}

\subsection{OS}\label{os}

\begin{verbatim}
import os
os.chdir(path) # change dir
\end{verbatim}

\subsection{Python:}\label{python}

Concatenation using \texttt{+}

\begin{verbatim}
"hello" + " " + "world"  # 'hello world'
[3,4,5] + [4,9,7] # [3,4,5, 4,9,7]
\end{verbatim}

String formatting using \texttt{string.format()}

\begin{verbatim}
print('Total is: {0}'.format(total))
\end{verbatim}

\texttt{zip} to loop over a sequence of tuples

\begin{verbatim}
for value, weight in zip([2,3,19],
                         [0.2,0.3,0.5]):
    total += weight * value
\end{verbatim}

\subsection{Numpy}\label{numpy}

\subsubsection{Numpy functions:}\label{numpy-functions}

\texttt{np.sum(x)}, \texttt{np.sqrt(x)} (entry wise). \texttt{x**2}
(entry wise power), \texttt{np.corrcoef(x,y)} (find the correlation
coefficient of array \texttt{x} and array \texttt{y})

\texttt{np.mean(axis=None)}: axis could be \texttt{None} (all entries),
\texttt{0}(along row), \texttt{1}(along column)

\texttt{np.var(x,\ ddof=0)}, \texttt{np.std(x,\ ddof=0)}, \# Note both
\texttt{np.var} and \texttt{np.std} accepts an argument \texttt{ddof},
the divisor is \texttt{N-ddof}.

\texttt{np.linspace(-np.pi,\ np.pi,\ 50)} \# start, end, number of
points 50

\texttt{np.multiply.outer(row,col)} \# calculate the product over the
mesh with vectors \texttt{row} and \texttt{col}.

\texttt{np.zeros(shape\ or\ int,\ dtype)} \#eg:
\texttt{np.zeros(5,bool)}

\texttt{np.ones(Boston.shape{[}0{]})}

\texttt{np.all(x)}, \texttt{np.any(x)}: check if all or any entry of
\texttt{x} is true.

\texttt{np.unique(x)}: find unique values in \texttt{x}.
\texttt{np.isnan(x)}: return a boolean array of \texttt{len(x)}.
\texttt{np.isnan(x).mean()}: find the percentage of \texttt{np.nan}
values in \texttt{x}.

\subsubsection{Array Slicing and
indexing}\label{array-slicing-and-indexing}

\texttt{np.arange}(start, stop, step) \# numpy version of \texttt{range}

\texttt{x{[}slice(3:6){]}} \# equivalent to \texttt{x{[}3:6{]}}

Indexing an array using \texttt{{[}row,\ col{]}} format. If \texttt{col}
is missing, then index the entire rows. \texttt{len(row)} must be equal
to \texttt{len(col)}. Otherwise use iterative indexing or use
\texttt{np.ix\_(x\_idx,\ y\_idx)} function, or use Boolean indexing, see
below.

\begin{verbatim}
A[1,2]: index entry at row 1 and col 2 (recall Python index start from 0)
A[[1,3]] # row 1 and 3. Note the outer [] is considered as the operator, so only row indices are provided. 
A[:,[0,2]] # cols 0 and 2
A[[1,3], [0,2,3]] # entry A[1,0] and A[3,2]
A[1:4:2, 0:3:2] # entries in rows 1 and 3, cols 0 and 2
A[[1,3], [0,2,3]] # syntax error
# instead one can use the following two methods 
A[[1,3]][:,[0,2]] # iterative subsetting
A[np.ix_([1,3],[0,2,3])] # use .ix_ function to create an index mesh
A[keep_rows, keep_cols] # keep_rows, keep_cols are boolean arrays of the same length of rows or cols, respectively
A[np.ix_([1,3],keep_cols)] # np.ix_()can be applied to mixture of integer array and boolean array
\end{verbatim}

\subsubsection{Random numbers and
generators}\label{random-numbers-and-generators}

\begin{verbatim}
np.random.normal(loc=0.0, scale=1.0,size=None) # size can be an integer or a tuple.
# 
rng = np.random.default_rng(1303) # set random generator seed
rng.normal(loc=0, scale=5, size=2) # 
rng.standard_normal(10) # standard normal distribution of size 10
rng.choice([0, np.nan], p=[0.8,0.2], size=A.shape)
\end{verbatim}

\subsubsection{Numpy array atributes}\label{numpy-array-atributes}

\texttt{.dtype}, \texttt{.ndim}, \texttt{.shape}

\subsubsection{Numpy array methods}\label{numpy-array-methods}

\texttt{x.sum(axis=None)} (equivalent to \texttt{np.sum(x)}),
\texttt{x.T} (transpose),

\texttt{x.reshape((2,3))} \# x.reshape() is a reference to x.

\texttt{x.min()}, \texttt{x.max()}

\subsection{Graphics}\label{graphics}

\subsubsection{2-D figure}\label{d-figure}

\begin{verbatim}
# Using the subplots + ax methods
fig, ax = subplots(nrows=2, ncols=3, figsize=(8, 8)) 
# explicitly name each axis in the grid 
fig, ((ax1, ax2), (ax3, ax4)) = plt.subplots(nrows=2, ncols=2, figsize=(10,10))

ax[0,1].plot(x, y,marker='o', 'r--', linewidth=3); #line plot. `;` suppresses the text output. pick ax[0,1] when there are  multiple axes
ax.plot([min(fitted),max(fitted)],[0,0],color = 'k',linestyle = ':', alpha = .3)
ax.scatter(x, y, marker='o'); #scatter plot
ax.scatter(fitted, residuals, edgecolors = 'k', facecolors = 'none')
ax.set_xlabel("this is the x-axis")
ax.set_ylabel("this is the y-axis")
ax.set_title("Plot of X vs Y");
axes[0,1].set_xlim([-1,1]) # set x_lim. similarly `set_ylim()`

fig = ax.figure  # get the figure object from an axes object
fig.set_size_inches(12,3) # access the fig object to change fig size (width, height)
fig # re-render the figure
fig.savefig("Figure.pdf", dpi=200); #save a figure into pdf. Other formats: .jpg, .png, etc
\end{verbatim}

\subsubsection{Contour and image}\label{contour-and-image}

\begin{verbatim}
fig, ax = subplots(figsize=(8, 8))
x = np.linspace(-np.pi, np.pi, 50)
y = x
f = np.multiply.outer(np.cos(y), 1 / (1 + x**2))
ax.contour(x, y, f, levels=None); # numbre of levels. if None, automatically choose
ax.imshow(f); # heatmap colorcoded by f
\end{verbatim}

\subsection{Pandas}\label{pandas}

\subsubsection{loading data}\label{loading-data}

\begin{verbatim}
pd.read_csv('Auto.csv') # read csv
pd.read_csv('Auto.data', 
            na_values =['?'], #specifying the na_values in the datafile. 
            delim_whitespace=True) # read whitespaced text file
pd.read_csv('College.csv', index_col=0) # use column `0` as the row labels 
\end{verbatim}

\subsubsection{Pandas Dataframe attributes and
methods}\label{pandas-dataframe-attributes-and-methods}

\begin{verbatim}
Auto.shape
Auto.columns # gets the list of column names
Auto.index #return the index (labels) objects
Auto['horsepower'].to_numpy() # convert to numpy array
Auto['horsepower'].sum()

Auto.dropna() # drop the rows containing na values. 
df.drop('B', axis=1, inplace=True) # drop a column 'B' inplace. 
#equivalent to df.drop(columns=['B'], inplace=True)
df.drop(index=['Ohio','Colorado']) #eqivalent to: df.drop(['Ohio','Colorado'], axis=0)
auto_df.drop(auto_df.index[10:86]) # drop rows with index[10:86] not including 86

Auto.set_index('name')# rename the index using the column 'name'.

pd.Series(Auto.cylinders, dtype='category') # convert the column `cylinders` to 'category` dtype
# the convertison can be done using `astype()` method
Auto.cylinders.astype('category')
Auto.describe() # statistics summary of all columns
Auto['mpg'].describe() # for selected columns

college.rename({'Unnamed: 0': 'College'}, axis=1): # change column name, 
# alternavie way
college_df.rename(columns={college_df.columns[0] : "College"}, inplace=True) #

college['Elite'] = pd.cut(college['Top10perc'],  # binning a column
                          [0,0.5,1],  #bin edges
                          labels=['No', 'Yes'],  # bin labels (names)
                          right=True,# True: right-inclusive (default) for each bin ( ]; False:rigth-exclusive 
                          )   
college['Elite'].value_counts() # frequency counts
auto.columns.tolist() # equivalent to  auto.columns.format() (rarely used)
\end{verbatim}

\subsubsection{Selecting rows and
columns}\label{selecting-rows-and-columns}

Select Rows:

\begin{verbatim}
Auto[:3] # the first 3 rows. 
Auto[Auto['year'] > 80] # select rows with boolean array
Auto_re.loc[['amc rebel sst', 'ford torino']] #label_based row selection
Auto_re.iloc[[3,4]] #integer-based row seleciton: rows 3 and 4 (index starting from 0)
\end{verbatim}

Select Columns

\begin{verbatim}
Auto['horsepower'] # select the column 'horsepower', resulting a pd.Series.
Auto[['horsepower']] #obtain a dataframe of the column 'horsepower'. 
Auto_re.iloc[:,[0,2,3]] # intger-based selection
auto_df.select_dtypes(include=['int16','int32']) # select columns by dtype
\end{verbatim}

Select a subset

\begin{verbatim}
Auto_re.iloc[[3,4],[0,2,3]] # integer-based 
Auto_re.loc['ford galaxie 500', ['mpg', 'origin']] #label-based 
Auto_re.loc[Auto_re['year'] > 80, ['weight', 'origin']] # mix bolean indexing with labels

Auto_re.loc[lambda df: (df['year'] > 80) & (df['mpg'] > 30),
            ['weight', 'origin']
           ]  # using labmda function with loc[]
\end{verbatim}

\subsubsection{Pandas graphics}\label{pandas-graphics}

Without using \texttt{subplots} to get axes and figure objects

\begin{verbatim}
ax = Auto.plot.scatter('horsepower', 'mpg') #scatter plot of 'horsepower' vs 'mpg' from the dataframe Auto
ax.set_title('Horsepower vs. MPG');
fig = ax.figure
fig.savefig('horsepower_mpg.png');

plt.gcf().subplots_adjust(bottom=0.05, left=0.1, top=0.95, right=0.95) #in percentage of the figure size. 
ax1.fig.suptitle('College Scatter Matrix', fontsize=35)
\end{verbatim}

Using \texttt{subplots}

\begin{verbatim}
fig, axes = subplots(  ncols=3, figsize=(15, 5))
Auto.plot.scatter('horsepower', 'mpg', ax=axes[1]);
Auto.hist('mpg', ax=ax);
Auto.hist('mpg', color='red', bins=12, ax=ax); # more customized 
\end{verbatim}

Boxplot using \texttt{subplots}

\begin{verbatim}
Auto.cylinders = pd.Series(Auto.cylinders, dtype='category') # needs to convert the `cylinders` column to categorical dtype
fig, ax = subplots(figsize=(8, 8))
Auto.boxplot('mpg', by='cylinders', ax=ax);
\end{verbatim}

Scatter matrix

\begin{verbatim}
pd.plotting.scatter_matrix(Auto); # all columns
pd.plotting.scatter_matrix(Auto[['mpg',
                                 'displacement',
                                 'weight']]);  # selected columns
                                 
                                 
#Alternatively with sns.pairplot

\end{verbatim}

Sns Graphic

\begin{verbatim}
# Scatter matrix
ax1 = sns.pairplot(college_df[college_df.columns[0:11]])

# Boxplot
sns.boxplot(ax=ax, x="Private", y="Outstate", data=college_df)
\end{verbatim}

\bookmarksetup{startatroot}

\chapter{Chapter 3: Linear
Regression}\label{chapter-3-linear-regression}

Linear regression is a simple supervised learning assuming a linear
relation between \(Y\) and \(X\). When there is only one predictor, it's
a \textbf{simple linear regression}. When there are more than one
predictors, it's called \textbf{multiple linear regression}. Note
\emph{multivariate regression} refer to the \(Y\) variable is a vector.

\section{Simple Linear Regression}\label{simple-linear-regression}

Assumes the \emph{population regression line} model \[
Y = \beta_0 + \beta_1 X +\epsilon, 
\] where, \(\beta_0\) is the \emph{expected} value of \(Y\) when
\(X=0\), and \(\beta_1\) is the \emph{average} change in \(Y\) with a
one-unit increase in \(X\). \(\epsilon\) is a ``catch all'' error term.

After training using the training data, we can obtain the parameter
estimates \(\hat{\beta}_0\) and \(\hat{\beta}_1\). The we can obtain the
prediction for \(x\) given by the \emph{least square line}: \[
\hat{y} = \hat{\beta}_0 + \hat{\beta}_1 x
\] The error at a data point \(x_i\) is given by
\(e_i = y_i -\hat{y}_i\), and the \emph{residual sum of squares} (RSS)
is \[
\text{RSS} =e_1^2+\cdots +e_n^2. 
\] One can use the least square approach to minimize RSS to obtain \[
\hat{\beta}_1 =\frac{(x_i-\bar{x})(y_i-\bar{y})}{\sum_{i=1}^n(x_i-\bar{x})^2}=r_{xy}\frac{\sigma_y}{\sigma_x}
\] \[
\hat{\beta}_0= \bar{y}-\hat{\beta}_1 \bar{x}
\] where, \(\bar{y}=\frac{1}{n}\sum_{i=1}^n y_i\) and
\(\bar{x}=\frac{1}{n}\sum_{i=1}^n x_i\). If we assume the data matrix
\(X\) is demeaned, then \(\hat{beta}_0=\bar{y}\). and the correlation
\begin{equation}\phantomsection\label{eq-correlation-rxy}{
r_{xy} = \frac{\text{cov}(x,y)}{\sigma_x\sigma_y}=\frac{(x_i-\bar{x})(y_i-\bar{y})}{\sqrt{\sum_{i=1}^n(x_i-\bar{x})^2}\sqrt{\sum_{i=1}^n(y_i-\bar{y})^2}}.
}\end{equation} is the normalized covariance. Note
\(-1\le r_{xy} \le 1\). When there is no intercept, that is
\(\beta_0=0\), then \[
\hat{y}_i=x_i \hat{\beta}=\sum_{i=1}^n a_i y_i
\] where, \[
\hat{\beta} =\frac{\sum_{i=1}^n x_iy_i}{ \sum_{i=1}^{n} x_i^2}
\] That is, the fitted values are linear combinations of the response
values when there is no intercept.

\subsection{Assessing the accuracy of the
coefficients}\label{assessing-the-accuracy-of-the-coefficients}

Let \(\sigma^2=\text{Var}(\epsilon)\), that is, \(\sigma^2\) is the
variance of \(Y\), (estimated by
\(\sigma^2\approx =\text{RSE} =\text{RSS}/(n-p-1)\). ) Assume each
observation have \emph{common variance} (homoscedasticity) and are
\emph{uncorrelated}, then the standard errors under repeated sampling \[
(\text{SE}[\hat{\beta}_1])^2 = \frac{1}{\sigma^2_x}\cdot \frac{\sigma^2}{n}
\] \[
(\text{SE}[\hat{\beta}_0])^2 = \left[1+ \frac{\bar{x}^2}{\sigma^2_x} \right]\cdot \frac{\sigma^2}{n}
\]

\begin{itemize}
\item
  when \(x_i\) are more spread out (with large \(\sigma_x^2\)), then
  \(\text{SE}[\hat{\beta}_1]\) is small. This is because there are more
  \emph{leverage} (of \(x\) values) to estimate the slope.
\item
  when \(\bar{x} =0\) , then
  \(\text{SE}[\hat{\beta}_0] = \text{SE}[\bar{y}]\). In this case,
  \(\hat{\beta}_0 = \bar{y}\).
\end{itemize}

Standard errors are used to construct CI and perform hypothesis test for
the estimated \(\hat{\beta}_0\) or \(\hat{\beta}_1\). Under the
assumption of \textbf{Gaussian error}, One can construct the CI of
significance level \(\alpha\) (e.g., \(\alpha=0.05\)) as \[
\hat{\beta}_j = [\hat{\beta}_j- t_{1-\alpha/2,n-p-1}\cdot \text{SE}[\hat{\beta}_j], \hat{\beta}_j+ t_{1-\alpha/2,n-p-1} \cdot \text{SE}[\hat{\beta}_j]  ]
\] Where \(j=0, 1\). Large interval including zero indicates \(\beta_j\)
is not statistically significant from 0. When \(n\) is sufficient large,
\(t_{0.975,n-p-1} \approx 2\). With the standard errors of the
coefficients, one can also perform \textbf{hypothesis test} on the
coefficients. For \(j=0,1\),

\[H_0: \beta_j=0\] \[H_A: \beta_j\ne 0\] The \(t\)-statistic of degree
\(n-p-1\), given by \[
t = \frac{\hat{\beta}_j - 0}{\text{SE}[\hat{\beta}_j]}
\] shows how far away \(\hat{\beta}_j\) is away from zero, normalized by
its error \(\text{SE}[\hat{\beta}_j]\). One can then compute the
\(p\)-value corresponding to this \(t\) and test the hypothesis. Small
\(p\)-value indicates \textbf{strong} relationship.

\section{Multiple Linear Regression}\label{multiple-linear-regression}

\[
Y= \beta_0 + \beta_1X_1 +\cdots + \beta_pX_p + \epsilon.
\] The estimate of the coefficients \(\hat{\beta_j}\),
\(j\in \mathbb{Z}_{p+1}\) are found by using the same least square
method to minimize RSS. we interpret \(\beta_j\) as the \emph{expected}
(average) effect on \(Y\) with one unit increase in \(X_j\),
\textbf{holding all other predictors fixed}. This interpretation is
based on the assumptions that \emph{the predictors are uncorrelated}, so
\emph{each predictor can be estimated and tested separately}. When there
are correlations among predictors, the variance of all coefficients
tends to increase, sometimes dramatically, and the previous
interpretation becomes hazardous because when \(X_j\) changes,
everything else changes.

\subsection{\texorpdfstring{\textbf{Model
Assumption}}{Model Assumption}}\label{model-assumption}

\begin{itemize}
\item
  linearity: \(Y\) is linear in \(X\). The change in Y associated with
  one unit of change in \(X_j\) is constant, regardless of the value of
  \(X_j\). This can be examined visually by plotting the \emph{residual
  plot} (\(e_i\) vs.~\(x_i\) for \(p=1\) or \(e_i\) vs \(\hat{y}_i\) for
  multiple regression). If the linear assumption is true, then the
  residual plot should not exhibit obvious pattern. If there is a
  nonlinear relationship suggested by by the residual plot, then a
  simple approach is to include transformed \(X\), such as \(\log X\),
  \(\sqrt{X}\), or \(X^2\).
\item
  additive: The association between \(X_j\) and \(Y\) is independent of
  other predictors.
\item
  Errors \(\epsilon_i\) are uncorrelated. This means \(\epsilon_i\)
  provides no information for \(\epsilon_{i+1}\). Otherwise (for
  example, frequently observed in a time series, where error terms are
  positively correlated, and \emph{tracking} is observed in the
  residuals, i.e., adjacent error terms take similar values), the
  estimated standard error will tend to be underestimated, hence leading
  less confidence in the estimated model.
\item
  Homoscedasticity: \(\text{Var}(\epsilon_i) =\sigma^2\). The error
  terms have constant variance. If not (heteroscedasticity), one may use
  transformed \(Y\), such as \(\sqrt{Y}\), or \(\log(Y)\) to mitigate
  this; or use \emph{weighted least squares} if it's known that for
  example \(\sigma_i^2=\sigma^2/n_i\).
\item
  Non-colinearity: two variables are colinear if they are highly
  correlated with each other. Co-linearity causes a great deal of
  \emph{uncertainty} in the coefficient estimates, that is, reducing the
  accuracy of the coefficient estimates, thus cause the standard error
  of \(\beta_j\) to grow, and hence smaller \(t\)-statistic. As a
  result, we may fail to reject \(H_0: \beta_j=0\). This in turn means
  the power of Hypothesis test, the probability of correctly detecting a
  \emph{non-zero} coefficient is reduced by colinearity. To detect
  colinearity,

  \begin{itemize}
  \item
    use the correlation matrix of predictors. Large value of the matrix
    in absolute value indicates highly correlated variable pairs. But
    this approach cannnot detect \emph{multicolinearity}.
  \item
    Use VIF (Variance inflation factor, VIF \(\ge 1\)) to detect
    multicolinearity. It is possible for colinearity exists between
    three or more variables even if no pair of variables has a
    particularly high correlation. This is the \emph{multicolinearity}
    situation.
  \end{itemize}

  VIF is the ratio of the variance of \(\hat{\beta}_j\) when fitting the
  full model divided by the variance of \(\hat{\beta}_j\) if fit on its
  own. It can be calculated by \[
    \text{VIF}(\hat{\beta}_j) =\frac{1}{1-R^2_{X_j|X_{-j}}}
    \] Where \(R^2_{X_j|X_{-j}}\) is the \(R^2\) from a regression of
  \(X_j\) onto all of the other predictors. A VIF value exceeds 5 or 10
  (i.e., \(R^2_{X_j|X_{-j}}\) close to 1) indicates colinearity.

  To remedy a colinearity problem:

  \begin{itemize}
  \tightlist
  \item
    drop a redundant variable (variables with colinearity should have
    similar VIF values. )
  \item
    Combine the colinear variables into a single predictor, e.g., taking
    the average of the standardized versions of those variables.
  \end{itemize}
\end{itemize}

\textbf{Claims of causality should be avoided for observational data}.

\subsection{Assessing existence of linear
relationship}\label{assessing-existence-of-linear-relationship}

\begin{itemize}
\tightlist
\item
  test Hypothesis (test if there is a linear relationship between the
  response and predictors) \[
  H_0: \beta_1=\beta_2=\cdots = \beta_p=0
  \] \[
  H_a: \text{at least one } \beta_j \text{ is non-zero.}
  \] using \(F\)-statistic \[
  F=\frac{\text{SSB/df(B)}}{\text{SSW/df(W)}}=\frac{(\text{TSS}-\text{RSS})/p}{\text{RSS}/(n-p-1)}\sim F_{p,n-p-1}
  \] If \(H_0\) is true, \(F\approx 1\); if \(H_a\) is true, \(F>>1\).
  \(F\)-statistic adjust with \(p\). Note that one \textbf{cannot
  conclude} if an individual \(t\)-statistic is significant, then there
  is at least one predictor is related to the response, especially when
  \(p\) is large. This is related to \emph{multiple testing}. The reason
  is that when \(p\) is large, there is \(\alpha\) (eg 5\%) chance that
  a predictor will have a small \(p\)-value by chance. When \(p>n\),
  \(F\)-statistic cannot be used.
\end{itemize}

If the goal is to test that a particular subset of \(q\) of the
coefficients are zero, that is, (for convenience, we put the \(q\)
variables chosen at the end of the variabale list)
\begin{equation}\phantomsection\label{eq-subset-hypothesis}{
H_0: \beta_{p-q+1} = \beta_{p-q+2}=\cdots = \beta_p=0
}\end{equation} In this case, use \[
F = \frac{(\text{RSS}_0-\text{RSS})/q}{\text{RSS}/(n-p-1)}\sim F_{q,n-p-1}
\] where, \(\text{RSS}_0\) is the residual sum of squares of a second
model that uses all variables \emph{except} those last \(q\) variables.
When \(q=1\), \(F\)-statistic in Equation~\ref{eq-subset-hypothesis} is
the square of the \(t\)-statistic of that variable. The \(t\)-statistic
reported in a regression model gives the \emph{partial effect} of adding
that variable, while holding other variables fixed.

\subsection{Assess the accuracy of the future
prediciton}\label{assess-the-accuracy-of-the-future-prediciton}

\begin{itemize}
\item
  confidence interval: Indicate how far away \(\hat{Y}=\hat{f}(X)\) is
  from the population average \(f(X)\) because the coefficients
  \(\hat{\beta}_{j}\) are estimated, It quantifies \emph{reducible
  error} around the predicted average response \(\hat{f}(X)\), does-not
  include \(\epsilon\).
\item
  prediction interval: Indicate how far away \(\hat{Y}=\hat{f}(X)\) is
  from \(Y\). predict an individual response
  \(Y\approx \hat{f}(X)+\epsilon\). Prediction interval is always wider
  than the confidence interval, because it includes \emph{irreducible
  error} \(\epsilon\).
\end{itemize}

\subsection{Assessing the overall accuracy of the
model}\label{assessing-the-overall-accuracy-of-the-model}

\begin{itemize}
\item
  RSE. To this end, first define the \emph{lack of fit} measure
  \textbf{Residual Standard Error} \[
  \text{RSE} = \sqrt{\frac{1}{n-p-1}\text{RSS}} = \sqrt{\frac{1}{n-p-1}\sum_{i=1}^n(y_i-\hat{y}_i)^2} \approx \sigma=\sqrt{\text{Var}(\epsilon)}
  \] It is the \emph{average amount} in \(\hat{Y}\) that a response
  deviates from the \emph{true regression line} (\(\beta_0+\beta_1 X\)).
  Note, RSE can increase with more variables if the decrease of RSS
  doesnot offset the increase of \(p\).
\item
  Approach 2: Using \emph{R-squared} (fraction of variance in \(Y\)
  explained by \(X\)), which is independent of of the scale of \(Y\),
  and \(0\le R^2 \le 1\): \[
  R^2 =\frac{\text{TSS}-\text{RSS}}{\text{TSS}} = 1-\frac{\text{RSS}}{\text{TSS}}
  \] where, \(\text{TSS}=\sum_{i=1}^n(y_i- \bar{y})\). When \(R^2\) is
  near 0 indicates that 1) either the linear model is wrong 2) or th
  error variance \(\sigma^2\) is high, or both. \(R^2\) measures the
  linear relationship between \(X\) and \(Y\). If computed on the
  training set, when adding more variables, the RSS always decrease,
  hence \(R^2\) will always increase.
\end{itemize}

For simple linear regression, \(R^2=r_{xy}^2\), where the \emph{sample
correlation} measures the linear relationship between variables \(X\)
and \(Y\). See the formula \(r_{xy}\) above
Equation~\ref{eq-correlation-rxy}. For multiple linear regression,
\(R^2=(\text{Cor}(Y, \hat{Y}))^2\). The fitted linear model maximizes
this correlation among all possible linear models.

\section{Model Selection/Variable Selections: balance training errors
with model
size}\label{model-selectionvariable-selections-balance-training-errors-with-model-size}

\begin{itemize}
\item
  \textbf{All subsets (best subsets) regression}: compute the least
  square fit for all \(2^p\) possible subsets and then choose among them
  based on certain criterion that balance training error and model size
\item
  \textbf{Forward selection}: Start from the \emph{null model} that only
  contains \(\beta_0\). Then find the best model containing one
  predictor that minimizing RSS. Denote the variable by \(\beta_1\).
  Then continue to find the best model with the lowest RSS by adding one
  variable from the remaining predictors, and so on. Continue until some
  stopping rule is met: e.g., when all remaining variables have a
  \(p\)-value greater than some threshold.
\item
  \textbf{Backward selection}: start with all variables in the model.
  Remove the variable with the largest \(p\)-value (least statistically
  significant). The new \((p-1)\) model is fit, and remove the variable
  with the largest \(p\)-value. Continue until a stopping rule is
  satisfied, e.g., all remaining variables have \(p\)-value less than
  some threshold.
\item
  \textbf{Mixed selection}: Start with forward selection. Since the
  \(p\)-value for variables can become larger as new predictors are
  added, at any point if the \(p\)-value of a variable in the model
  rises above a certain threshold, then remove that variable. Continue
  to perform these forward and backward steps until all variables in the
  model have a sufficiently low \(p\)-value, and all variables outside
  the model would have a large \(p\)-value if added to the model.

  Backward selection cannot be used if \(p>n\). Forward selection can
  always be used, but might include variables early that later become
  redundant. Mixed selection can remedy this problem.
\item
  \textbf{others} (Chapter 6): including Mallow's \(C_p\), AIC (Akaike
  Informaton Criterion), BIC, adjusted \(R^2\), Cross-validation, test
  set performance.
\item
  \textbf{not valid}: we could look at individual \(p\)-values, but when
  the number of variables \(p\) is large, we likely to make a false
  discoveries.
\end{itemize}

\section{Handle categorical variables (factor
variables)}\label{handle-categorical-variables-factor-variables}

For a categorical variable \(X_i\) with \(m\) levels, create one fewer
dummy variables (\(x_{ij}, 1\le j \le m-1\))\textgreater. The level with
no dummy variable is called the \emph{baseline}. The coefficient
corresponding to a dummy variable is the expected difference in change
in \(Y\) when compared to the baseline, while holding other predictors
fixed.

\section{Adding non-linearity}\label{adding-non-linearity}

\subsection{Modeling interactions
(synergy)}\label{modeling-interactions-synergy}

When two variables have interaction, then their product \(X_iX_j\) can
be added into the regression model, and the product maybe considered as
a single variable for inference, for example, compute its SE,
\(t\)-statistics, \(p\)-value, Hypothesis test, etc.

If we include an interaction in a model, then the \textbf{Hierarchy
principle} should be followed: always include the main effects, even if
the \(p\)-values associated with their coefficients are not significant.
This is because without the main effects, the interactions are hard to
interpret, as they would also contain the main effect.

\subsection{Adding terms of transformed
predictors}\label{adding-terms-of-transformed-predictors}

\begin{enumerate}
\def\labelenumi{\arabic{enumi})}
\tightlist
\item
  \emph{Polynomial regression}: Add a term involving \(X_i^k\) for some
  \(k>1\).
\item
  other forms: Adding root or logarithm terms of the predictors.
\end{enumerate}

\section{\texorpdfstring{Outliers (Unusual \(y_i\) that is far from
\(\hat{y}_i\))}{Outliers (Unusual y\_i that is far from \textbackslash hat\{y\}\_i)}}\label{outliers-unusual-y_i-that-is-far-from-haty_i}

It is typical for an outlier that does not have an unusual predictor
value (with low levarage) to have little effect on the least squares
fit, but it will increase RSE, hence deteriorate CI, \(p\)-value and
\(R^2\), thus affecting interpreting the model.

An outlier can be identified by computing the
\[\text{studentized residual}=\frac{e_i}{\text{RSE}_i}\] A studentized
residual great than 3 may be considered as an outlier.

\section{\texorpdfstring{High leverage points (unusual
\(x_i\))}{High leverage points (unusual x\_i)}}\label{high-leverage-points-unusual-x_i}

High leverage points tend to have sizeable impact on the regression
line. To quantify the observation's leverage, one needs to compute the
\textbf{leverage statistic}
\[h_i = \frac{1}{n}+ \frac{(x_i-\bar{x})^2}{\sum_{j=1}^n (x_j-\bar{x})^2}.\]
\(1/n \le h_i\le 1\) and \(\text{Ave}(h_i)=(p+1)/n\). A large value of
this statistic (for example, great than \((p+1)/n\)) indicates an
observation with high leverage. The leverage \(1/n\le h_i\le 1\),
reflects the amount an observation influences its own fit.

\section{Compared to KNN Regression}\label{compared-to-knn-regression}

KNN regression is a non-parametric method that makes prediction at
\(x_0\) by taking the average in a \(K\)-point neightborhood \[
\hat{f}(x_0) = \frac{1}{K}\sum_{x_i \in \mathcal{N}_{x_0}}{y_i}
\] A small value of \(K\) provides more flexible model with low bias but
high variance while a larger value of \(K\) provides smoother fit with
less variance. An optimal value of \(K\) depend on the
\emph{bias-variance tradeoff}. For non-linear data set, KNN may provides
better fit than a linear regression model. However, in higher dimension
(e.g., \(p\ge 4\)), even for nonlinear data set, KNN may perform much
inferior to linear regression, because of the \textbf{curse of
dimensionality}, as the \(K\) observations that are nearest to \(x_0\)
may in fact far away from \(x_0\).

\section{Homework (* indicates
optional):}\label{homework-indicates-optional-1}

\begin{itemize}
\tightlist
\item
  Conceptual: 1--6
\item
  Applied: 8--15. at least one.
\end{itemize}

\section{Code Gist}\label{code-gist-1}

\subsection{Python}\label{python-1}

\begin{verbatim}
dir() # provides a list of objects at the top level name space
dir(A) # display addtributes and methods for the object A
' + '.join(X.columns) # form a string by joining the list of column names by "+"
\end{verbatim}

\subsection{Numpy}\label{numpy-1}

\begin{verbatim}
np.argmax(x) # identify the location of the largest element
np.concatenate([x,y],axis=0) # concatenate two arrays x and y. 
\end{verbatim}

\subsection{Pandas}\label{pandas-1}

\begin{verbatim}
X = pd.DataFrame(data=X, columns=['a','b'])

pd.DataFrame({'intercept': np.ones(Boston.shape[0]),
                  'lstat': Boston['lstat']}) # make a dataframe using a dictionary
Boston.columns.drop('medv','age') # drop the elements 'medv' and 'age' from the list of column names

pd.DataFrame({'vif':vals},
                   index=X.columns[1:]) # form a df by specifying index labels

X.values  # Convert dataframe X to numpy array
X.to_numpy() # recommended to replace the above method
DataFrame.corr(numeric_only=True) # correlations between columns 
x.sort_values(ascending=False)
pd.to_numeric(auto_df['horsepower'], errors='coerce') # if error, denote it by "NaN".
auto_df.dropna(subset= ['horsepower', 'mpg',], inplace=True) # looking for NaN in the columns in `subset`, otherwise, all columns

auto_df.drop('name', axis=1, inplace=True)

left2.join(right2, how="left") #join two databases by index. 
left1.join(right1, on="key") # left-join by left1["key"] and the index of right1. 
pd.concat([s1, s4], axis="columns", join="outer")
\end{verbatim}

\subsection{Graphics}\label{graphics-1}

\begin{verbatim}
xlim = ax.get_xlim() # get the x_limit values xlim[0], xlim[1]
ax.axline() # add a line to a plot
ax.axhline(0, c='k', ls='--'); # horizontal line
line, = ax.plot(x,y,label="line 1") # "line 1" is the legend
# alternatively the label can be set by 
line.set_label("line 1")
ax.scatter(fitted, residuals, edgecolors = 'k', facecolors = 'none')
ax.plot([min(fitted),max(fitted)],[0,0],color = 'k',linestyle = ':', alpha = .3)
ax.legend(loc="upper left", fontsize=25) # adding legendes
ax.annotate(i,xy=(fitted[i],residuals[i])) # annote at the xy position with i. 


plt.style.use('seaborn') # pretty matplotlib plots
plt.rcParams.update({'font.size': 16})
plt.rcParams["figure.figsize"] = (8,7)

plt.rc('font', size=10)
plt.rc('figure', titlesize=13)
plt.rc('axes', labelsize=10)
plt.rc('axes', titlesize=13)
plt.rc('legend', fontsize=8) # adjust legend globally
    
\end{verbatim}

\subsection{Using Sns}\label{using-sns}

\begin{verbatim}
sns.set(font_scale=1.25) # set font size 25% larger than default
sns.heatmap(corr, cbar=True, annot=True, square=True, fmt='.2f', annot_kws={'size': 10})
ax = sns.regplot(x=x, y=y)
\end{verbatim}

\subsection{Using Sklearn}\label{using-sklearn}

\begin{verbatim}
from sklearn.linear_model import LinearRegression
## Set the target and predictors
X = auto_df['horsepower']

### To get polynomial features
poly = PolynomialFeatures(interaction_only=True,include_bias = False)
X = poly.fit_transform(X)

y = auto_df['mpg']

## Reshape the columns in the required dimensions for sklearn
length = X.values.shape[0]
X = X.values.reshape(length, 1) #both X and y needs to be 2-D
y = y.values.reshape(length, 1)

## Initiate the linear regressor and fit it to data using sklearn
regr = LinearRegression()
regr.fit(X, y)
regr.intercept_
regr.coef_

pred_y = regr.predict(X)
\end{verbatim}

\subsection{Using statsmodels and
ISLP}\label{using-statsmodels-and-islp}

\begin{verbatim}
from ISLP import load_data
from ISLP.models import (ModelSpec as MS,
                         summarize,
                         poly)
                         
import statsmodels.api as sm
import statsmodels.formula.api as smf
from statsmodels.stats.outliers_influence \
     import variance_inflation_factor as VIF
from statsmodels.stats.anova import anova_lm

#Training
Boston = load_data("Boston") 
#hand-craft the design matrix X
X = pd.DataFrame({'intercept': np.ones(Boston.shape[0]), #design matrix. intercept column
                  'lstat': Boston['lstat']}) 
#the following is the preferred method to create X
design = MS(['lstat']) # specifying the model variables. Automatically add an intercept, adding "intercept=False" if no intercept. 
design = design.fit(Boston) # do intial computation as specified in the model object design by MS(), such as means or sd. This attached some statistics to the `design` object, and need to be applied to the new data for prediciton

X = design.transform(Boston) # apply the fitted transformation to the data to create X
#alternatiely, 
X = design.fit_transform(Boston) # this combines the .fit() and .transform() two lines

y = Boston['medv']
model = sm.OLS(y, X) # setup the model
model = smf.ols('mpg ~ horsepower', data=auto_df) # alternatively use smf formula, y~x
smf.ols("y ~ x -1" , data=df).fit() # "-1" not inclding the intercept
results = model.fit() # results is a dictionary:.summary(), .params 

results.summary()
results.params # coefficients
results.resid # reisdual array
results.rsquared # R^2
results.pvalues
np.sqrt(results.scale) # RSE
results.fittedvalues # fitted \hat(y)_i at x_i in the traning set


summarize(results) # summzrize() is from ISLP to show the esstial results from model.fit()

# Makding prediciton 
new_df = pd.DataFrame({'lstat':[5, 10, 15]})  # new test-set containing data where to make predicitons
newX = design.transform(new_df) # apply the same transform to the test-set
new_predictions = results.get_prediction(newX);
new_predictions.predicted_mean #predicted values
new_predictions.conf_int(alpha=0.05) #for the predicted values

new_predictions.conf_int(obs=True, alpha=0.05) # prediction intervals by setting obs=True

# Including an interaction term
X = MS(['lstat',
        'age',
        ('lstat', 'age')]).fit_transform(Boston) #interaction term ('lstat', 'age')

# Adding a polynomial term of higher degree
X = MS([poly('lstat', degree=2), 'age']).fit_transform(Boston) # Note poly is from ISLP, # adding deg1 and deg2 terms. by default poly creates ortho. poly. not including an intercept. 
# Given a qualitative variable, `ModelSpec()` generates dummy
variables automatically, to avoid collinearity with an intercept, the first column is dropped in the design matrix generated by 'ModelSpec()` by default.

# Compare nested models using ANOVA
anova_lm(results1, results3) # result1 is the result of linear model, an result3 is the result of a larger model

# Identify high leverage x
infl = results.get_influence() 
# hat_matrix_diag calculate the leverate statistics
np.argmax(infl.hat_matrix_diag) # identify the location of the largest levarage

# Calculate VIF
vals = [VIF(X, i)
        for i in range(1, X.shape[1])] #excluding column 0 because it's all 1's in X.
vif = pd.DataFrame({'vif':vals},
                   index=X.columns[1:])
vif # VIF exceeds 5 or 10 indicates a problematic amount of colinearity
\end{verbatim}

Useful Code Snippets

\begin{verbatim}
def abline(ax, b, m, *args, **kwargs):
    "Add a line with slope m and intercept b to ax"
    xlim = ax.get_xlim()
    ylim = [m * xlim[0] + b, m * xlim[1] + b]
    ax.plot(xlim, ylim, *args, **kwargs)
\end{verbatim}

\begin{verbatim}
# Plot scatter plot with a regression line
ax = Boston.plot.scatter('lstat', 'medv')
abline(ax,
       results.params[0],
       results.params[1],
       'r--',
       linewidth=3)
\end{verbatim}

\begin{verbatim}
# Plot residuals vs. fitted values (note, not vs x, therefore works for multiple regression)
ax = subplots(figsize=(8,8))[1]
ax.scatter(results.fittedvalues, results.resid)
ax.set_xlabel('Fitted value')
ax.set_ylabel('Residual')
ax.axhline(0, c='k', ls='--');

# Alternatively
sns.residplot(x=X, y=y, lowess=True, color="g", ax=ax)

# Plot the smoothed residuals~fitted by LOWESS
from statsmodels.nonparametric.smoothers_lowess import lowess
smoothed = lowess(residuals,fitted) # Note the order (y,x)
ax.plot(smoothed[:,0],smoothed[:,1],color = 'r')

# QQ plot for the residuas (obtain studentized residuals for identifying outliers)
import scipy.stats as stats
sorted_student_residuals = pd.Series(smf_model.get_influence().resid_studentized_internal)
sorted_student_residuals.index = smf_model.resid.index
sorted_student_residuals = sorted_student_residuals.sort_values(ascending = True)
df = pd.DataFrame(sorted_student_residuals)
df.columns = ['sorted_student_residuals']

#stats.probplot() #assess whether a dataset follows a specified distribution
df['theoretical_quantiles'] = stats.probplot(df['sorted_student_residuals'], dist = 'norm', fit = False)[0] 
    
x = df['theoretical_quantiles']
y = df['sorted_student_residuals']
ax.scatter(x,y, edgecolor = 'k',facecolor = 'none')
\end{verbatim}

\begin{verbatim}
# Plot leverage statistics
infl = results.get_influence()
ax = subplots(figsize=(8,8))[1]
ax.scatter(np.arange(X.shape[0]), infl.hat_matrix_diag)
ax.set_xlabel('Index')
ax.set_ylabel('Leverage')
np.argmax(infl.hat_matrix_diag) # identify the location of the largest levarage
\end{verbatim}

\bookmarksetup{startatroot}

\chapter{Chapter 4: Classification}\label{chapter-4-classification}

Given a feature vector \(X\) and a \emph{qualitative} (categorical)
response \(Y\) taking finite values in a set \(\mathcal{C}\), the
classification task is to build a classifier \(C(X)\) that takes an
input \(X\) and predicts its class \(Y=C(X)\in \mathcal{C}\). This is
often done by model \(P(Y=k|X=x)\) for each \(k\in \mathcal{C}\).

\section{Linear regression and
Classification}\label{linear-regression-and-classification}

\begin{itemize}
\tightlist
\item
  For a \emph{binary} classification, one can use linear regression and
  does a good job. In this case, the linear regression classifier is
  equivalent to LDA, because \[
  P(Y=1|X=x)= E[Y|X=x]
  \] However, linear regression may not represent a probability as it
  may give a value outside the interval \([0,1]\).
\item
  When there are more than two classes, linear regression is not
  appropriate, because any chosen coding of the \(Y\) variable imposes
  an ordering and fixed differences among categories, which may not be
  implied by the data set. If the coding changes, a dramatic function
  will be fitted, which is not reasonable. One should turn to
  \emph{multiclass logistic regression} or \emph{Discriminant Analysis}.
\end{itemize}

\section{Logistic Regression}\label{logistic-regression}

Logistic regression is a \emph{discriminative learning}, because it
directly calculates the conditional probability \(P(Y|X)\) to make
classification.

\subsection{Binary classification}\label{binary-classification}

with a single variable Logistic regression simply convert the linear
regression to probability by \[
p(X)=Pr(Y=1|X) =\frac{e^{\beta_0+\beta_1 X}}{1+ e^{\beta_0+\beta_1X}}.
\] Note the \emph{logit} or \emph{log odds} is linear \[
\log\left( \frac{p(X)}{1-p(X)}  \right) =\beta_0 +\beta_1 X.
\] Increasing \(X\) by one unit, changes the log odds by \(\beta_1\).
Equivalently, it multiplied the odds by \(e^{\beta_1}\). The rate of
change of \(p(X)\) is no longer a constant, but depends on the current
value of \(X\). Positive \(\beta_1\) implies increasing \(p(X)\), and
vice vesa.

The parameters are estimated by maximizing the \emph{liklihood} \[
\ell(\beta_0, \beta_1) =\prod_{i: y_i=1}p(x_i) \prod_{i:y_i=0}(1-p(x_i))
\] With the estimated parameters \(\hat{\beta_j}, j=0,1\), one can
calculate the probability \[
p(X)=Pr(Y=1|X) =\frac{e^{\hat{\beta_0}+\hat{\beta_1} X}}{1+ e^{\hat{\beta}_0+\hat{\beta}_1X}}
\]

\subsection{with multiple variables}\label{with-multiple-variables}

In this case, simply let the logit be a linear function of \(p\)
variables.

Note when there are multiple variables, it's possible to have variables
confounding (especially when two variables are correlated): the
coefficient of a variable may changes significantly or may change sign,
this is because the coefficient represents the rate of change in \(Y\)
of that variable when holding other variable constants. The coefficient
reflects the effect when other variables are hold constant, how the
variable affects \(Y\), and this effect may be different than when only
this variable is used in the model.

\begin{tcolorbox}[enhanced jigsaw, breakable, bottomtitle=1mm, left=2mm, toptitle=1mm, opacityback=0, titlerule=0mm, title=\textcolor{quarto-callout-note-color}{\faInfo}\hspace{0.5em}{Note}, colback=white, rightrule=.15mm, colframe=quarto-callout-note-color-frame, arc=.35mm, coltitle=black, bottomrule=.15mm, toprule=.15mm, colbacktitle=quarto-callout-note-color!10!white, leftrule=.75mm, opacitybacktitle=0.6]

One can include a nonlinear term such as a quadratic term in the logit
model, similar to a linear regression that includes a non-linear term.

\end{tcolorbox}

\subsection{Multi-class logistic regression (multinomial regression)
with more than two
classes}\label{multi-class-logistic-regression-multinomial-regression-with-more-than-two-classes}

in this case, we use the \emph{softmax} function to model \[
\text{Pr} (Y=k|X) =\frac{e^{\beta_{0k}+\beta_{1k}X_1+ \cdots + \beta_{pk}X_p}}{\sum_{\ell=1}^K e^{\beta_{0\ell}+\beta_{1\ell}X_1+ \cdots + \beta_{p\ell}X_p}} =a_k
\] for each class \(k\). Note \(\Sigma_k a_k=1\) and the cross-entropy
loss function is given by
\(-\log \ell(\beta)= -\Sigma_k \mathbb{1}_k \log a_k\), where \(\beta\)
represents all the parameters.

The log odds between \(k\)th and \(k'\)th classes equals \[
\log(\frac{\text{Pr}(Y=k|X=x)}{\text{Pr}(Y=k'|X=x)})=(\beta_{k0}-\beta_{k'0}) + (\beta_{k1}-\beta_{k'1}) + \cdots + (\beta_{kp}-\beta_{k'p})\]

\section{Discriminant Classifier: Approximating Optimal Bayes
Classifier}\label{discriminant-classifier-approximating-optimal-bayes-classifier}

Apply the Bayes Theorem, the model \[
\text{Pr}(Y=k|X=x)=\frac{\text{Pr}(X=x|Y=k)\cdot \text{Pr}(Y=k)}{\text{Pr}(X=x)}=\frac{\pi_k f_k(x)}{\sum_{\ell =}^ K \pi_{\ell}f_\ell(x)}
\] where \(\pi_k=\text{Pr(Y=k)}\) is the \emph{marginal} or \emph{prior}
probability for class \(k\), and \(f_k(x)=\text{Pr}(X=x|Y=k\)) is the
\emph{density} for \(X\) in class \(k\). Note the denominator is a
\emph{normalizing constant}. So when making decisions, effectively we
compare \(\pi_kf_k(x)\), and assign \(x\) to a class \(k\) with the
largest \(\pi_kf_k(x)\).

Discriminant uses the full liklihood \(P(X,Y)\) to calculate \(P(Y|X)\)
to make a classification, so it's known as \emph{generative learning}.

\begin{itemize}
\tightlist
\item
  when \(f_k\) is chosen as a normal distribution with constant variance
  (\(\sigma^2\)) for \(p=1\) or correlation matrix \(\Sigma\) for
  \(p>1\), this leads to the LDA. For \(p=1\), the \emph{discriminant
  score} is given by \[
  \delta_k(x) = x\cdot \frac{\mu_k}{\sigma^2}-\frac{\mu_k^2}{2\sigma^2}+\log (\pi_k)
  \] when \(K=2\) and \(\pi_1=\pi_2=0.5\)m then the \emph{decision
  boundary} is given by \[
  x=\frac{\mu_1+\mu_2}{2}. 
  \] When \(p\ge 2\), assume that \(X=(X_1, X_2, \cdots, X_p)\) is drawn
  from a multivariate Gaussian distribution \(X \sim N(\mu_k, \Sigma)\),
  with a class-specific mean vector and a a common variance matrix. \[
  \delta_k(x) =x^T\Sigma^{-1}\mu_k-\frac{1}{2}\mu_k^T\Sigma^{-1}\mu_k +\log \pi_k=c_{k0}+c_{k1}x_1+\cdots +c_{kp}x_p.
  \] The score function (posterior probability) is \emph{linear} in
  \(x\). With \(\hat{\delta}_k(x)\) for each \(k\), it can be converted
  to the class probability by the \emph{softmax} function \[
  \hat{\text{Pr}}(Y=k|X=x)=\frac{e^{\hat{\delta}_k(x)}}{\sum_{\ell=1}^K e^{\hat{\delta}_{\ell}(x)}}
  \] The \(\pi_k\), \(\mu_k\) and \(\sigma\) are estimate the follwing
  way: \[
  \hat{\pi}_k =\frac{n_k}{n}
  \] \[
  \hat{\mu}_k = \frac{1}{n_k} \sum_{i:y_i=k} x_i
  \] \[
  \hat{\sigma}^2 = \frac{1}{n-K}\sum_{k=1}^K \sum_{i:y_i=k}(x_i-\hat{\mu}_k)^2=\sum_{k=1}^{K}\frac{n_k-1}{n-K}\hat{\sigma}^2_k
  \] where
  \(\hat{\sigma}_k^2=\frac{1}{n_k-1} \sum_{i:y_i=k}(x_i-\hat{\mu}_k)^2\)
  is the estimated variance for the \(k\)-th class.
\end{itemize}

\begin{tcolorbox}[enhanced jigsaw, breakable, bottomtitle=1mm, left=2mm, toptitle=1mm, opacityback=0, titlerule=0mm, title=\textcolor{quarto-callout-note-color}{\faInfo}\hspace{0.5em}{Note}, colback=white, rightrule=.15mm, colframe=quarto-callout-note-color-frame, arc=.35mm, coltitle=black, bottomrule=.15mm, toprule=.15mm, colbacktitle=quarto-callout-note-color!10!white, leftrule=.75mm, opacitybacktitle=0.6]

One can include a nonlinear term such as a quadratic term in the LDA
model, similar to a linear regression that includes a non-linear term.

\end{tcolorbox}

\begin{itemize}
\item
  when each class chooses a different \(\Sigma_k\), then it's QDA. It
  assumes an observation from the \(k\)-th class is
  \(X\sim N(\mu_k, \Sigma_k)\).The score function has a \emph{quadratic}
  term \[
  \delta_k(x)=-\frac{1}{2}(x-\mu_k)^T\Sigma_k^{-1}(x-\mu_k)+\log \pi_k -\frac{1}{2}\log |\Sigma_k|
  \] QDA has much more parameters \(Kp(p+1)/2\) to estimate compared to
  LDA (\(Kp\)), hence has higher flexibility and may lead to higher
  variance. When there are few training examples, LDA tend to perform
  better and reducing variance is crucial. When there is a large traning
  set, QDA is recommended as variance is not a major concern. LDA is a
  special case of QDA.
\item
  when the features are modeled independently, i.e., there is no
  association between the \(p\) predictors,
  \(f_k(x) = \prod_{j=1}^p f_{jk}(x_j)\), the method is \emph{naive
  Bayes}, and \(\Sigma_k\) are diagonal. Any classifier with a linear
  decision boundary is a special case of NB. So LDA is a special case of
  NB. To estimate \(f_kj\), one can

  \begin{itemize}
  \item
    assume that \(X_j|Y=k \sim N(\mu_{jk,\sigma^2_{jk}})\), that is, a
    class specific covariance but is diagonal. QDA's \(\Sigma_k\) is not
    diagonal. If we model \(f_{kj}(x_j)\sim N(\mu_{kj}+\sigma_j^2)\)
    (Note \(\sigma_j^2\) is shared among clases), In this case NB is a
    special case of LDA that has a diagonal \(\Sigma\) and
  \item
    use a non-parametric estimate such as histogram (or a smooth kernel
    density estimator) for the observations of the jth Predictor within
    each class.
  \item
    If \(X_j\) is qualitative, then one can simply count the proportion
    of training observations for the \(j\)th predictor corresponding to
    each class.
  \item
    Can applied to \emph{mixed} feature vectors (qualitative and
    quantitative). NB does not assume normally distributed predictors.
  \item
    Despite strong assumptions, performs well, especially when \(n\) is
    not large enough relative to \(p\), when estimating the joint
    distribution is difficult. It introduces some biases but reduces
    variance, leading to a classifier that works quite well as a result
    of bias-variance trade-off.
  \item
    Useful when \(p\) is very large.
  \item
    NB is a \emph{generalized additive model}.
  \item
    Neigher NB nor QDA is a special case of the other. Because QDA
    contains interaction term \(x_ix_j\), while NB is purely additive,
    in the sense that a function of \(x_i\) is added to a function of
    \(x_j\). Therefore, QDA potentially is a better fit when the
    interactions among predictors are important.
  \end{itemize}
\end{itemize}

\subsection{Why discriminant analysis}\label{why-discriminant-analysis}

\begin{itemize}
\tightlist
\item
  When the classes are well-separated, the parameter estimation of
  logistic regression is unstable, while LDA does not suffer from this
  problem.
\item
  if the data size \(n\) is small and the distribution of \(X\) is
  approximately normal in each of the classes, then LDA is more stable
  than logistic regression. Also used when \(K>2\).
\item
  when there are more than two classes, LDA provides low-dimensional
  views of the data hence popular. Specifically, when there are \(K\)
  classes, LDA can be viewed exactly in \(K-1\) dimensional plot. This
  is because it essentially classifies to the closest centroid, and they
  span a \(K-1\) dimensional plane.
\item
  For a two-class problem, the logit of \(p(Y=1|X=x\)) by LDA
  (generative learning) is a linear function in \(X\), the same as a
  logistic regression (discriminative learning). The difference lies in
  how the parameters are estimated. But in practice, they are similar.
\item
  LDA assumes the predictors follow a multivariable normal distribution
  with a shared \(\Sigma\) among classes. So when this assumption holds,
  we expect LDA performs better; and Logistic regress performs better
  when this asuumption does not hold.
\end{itemize}

\section{KNN}\label{knn}

KNN is a non-parametric method and doesnot assume a shape for the
decision boundary. KNN assign the class of popularity to \(X=x\) in a
\(K\)-neighborhood.

\begin{itemize}
\item
  KNN dominates LDA and Logistic Regression when the decision boundary
  is highly non-linear, provided \(n\) is large and \(p\) is small. As
  KNN breaks down when \(p\) is large.
\item
  KNN requires large \(n>>p\), this is because KNN is non-parametric and
  tends to reduce bias but increase variance.
\item
  When the decision boundary is non-linear but \(n\) is only modest and
  \(p\) is not very small, QDA may outperform KNN. This is because QDA
  provides a non-linear boundary while taking advantage of a parametric
  form, which means that if requires smaller size for accurate
  classification.
\item
  Unlike logistic regression, KNN does not tell which predictors are
  more importnat: We dont get a table of coefficients.
\item
  When the decision boundary is linear, LDA or logistic regression may
  perform better, when the boundary is moderately non-linear, QDA or NB
  may perform better; For a much more complicated decision boundary, KNN
  may perform better.
\end{itemize}

\section{Poisson Regression}\label{poisson-regression}

When \(Y\) is discrete and non-negative, a linear regression model is
not satisfactory, even with the transformation of \(\log (Y)\), because
\(\log\) does not allow \(Y=0\).

\begin{itemize}
\tightlist
\item
  Poisson Regression: typically used to model counts, \[
  \text{Pr}(Y=k)= \frac{e^{-\lambda}\lambda^k}{k!}, \qquad k=0,1,2, \cdots,
  \] where, \(\lambda = E(Y)= \text{Var}(Y)\). This means that if \(Y\)
  follows a Poissson distribution, the larger the mean of \(Y\), the
  larger its variance. Posisson regression can handle this when variance
  changes with mean, but linear regression cannot, because it assumes
  constant variance.
\end{itemize}

Assume \[
\log(\lambda(X_1, X_2, \cdots, X_p))=\beta_0+\beta_1X_1+\cdots +\beta_pX_p
\] Then one can use maximum likelihood \[
\ell(\beta_0, \beta_1, \cdots, \beta_p)=\prod_{i=1}^n \frac{e^{-\lambda(x_i)}\lambda(x_i)^{y_i}}{y_i!}
\] to estimate the parameters.

\begin{itemize}
\tightlist
\item
  Interpretation: An increase in \(X_j\) by one unit is associated with
  a change in \(E(Y)=\lambda\) by a \emph{factor} (percentage) of
  \(\exp(\beta_j)\).
\end{itemize}

\section{Generalized Linear Models
(GLM)}\label{generalized-linear-models-glm}

Perform a regression by modeling \(Y\) from a particular member of the
\emph{exponential family} (Gaussian, Bernoulli, Poisson, Gamma, negative
binomial), and then transform the mean of \(Y\) to a linear function.

\begin{itemize}
\item
  Use predictors \(X_1, \cdots, X_p\) to predict \(Y\). Assume \(Y\)
  conditional on \(X\) follow some distribution: For linear regression,
  assume \(Y\) follows a normal distribution; for logistic regression,
  assume \(Y\) follows a Bernoulli (multinomial distribution for
  multi-class logistic regression) distribution; For poisson
  distribution, assume \(Y\) follows a poisson distribution.
\item
  Each approach models the mean of \(Y\) as a function of \(X\) using a
  \emph{linking function} \(\eta\) to transform \(E[Y|X]\) to a linear
  function.

  \begin{itemize}
  \item
    for linear regression \[
    E(Y|X)= \beta_0+\beta_1 X_1+\cdots +\beta_p X_p
    \] \(\eta(\mu) =\mu\)
  \item
    for logistic regression \[
    E(Y|X)=P(Y=1|X)=\frac{e^{\beta_0+\beta_1X_1+\cdots+\beta_pX_p}}{1+e^{\beta_0+\beta_1X_1+\cdots+\beta_pX_p}}
    \] \(\eta(\mu) = \log (\mu/(1-\mu))\)
  \item
    for Poisson regression \[
    E(Y|X) = \lambda(X) = e^{\beta_0+\beta_1X_1+\cdots+\beta_pX_p}
    \] \(\eta(\mu) = \log(\mu)\).
  \item
    Gamma regression and negative binomial regression.
  \end{itemize}
\end{itemize}

\section{Assessment of a classifier}\label{assessment-of-a-classifier}

\begin{itemize}
\item
  Confusion matrix
\item
  Overall error rate: equals to \[
  \frac{FP+FN}{N+P}
  \]
\item
  Class-specific performance: One can adjust the decision boundary
  (posterior probability threshold) to improve class specific
  performance at the expense of lowered overall performance.

  \begin{itemize}
  \item
    percentage of TP detected among all positives

    \[\text{sensitivity (recall, power)} = TPR = \frac{TP}{TP+FN}=\frac{TP}{P}= 1-\text{Type II error}=1-\beta\]
    this is equal to \(1- FNR\), where, FNR is The fraction of positive
    examples that are classified as negatives \[
    FNR = \frac{FN}{FN+TP}=\frac{FN}{P} 
    \]
  \item
    percentage of TN detected among all negatives
    \[\text{specificity}= TNR = \frac{TN}{TN+FP}=\frac{TN}{N}\] This is
    equal to \(1-FPR\), where, False positive rate (FPR): the fraction
    of negative examples (N) that are classified as positive: \[
    FPR=\frac{FP}{FP+TN}=\frac{FP}{N} = \text{Type I error} (\alpha) 
    \]
  \item
    ROC (receiver operating characteristic curve): plot true positive
    rate (TPR=1-Type II error) \textasciitilde{} false positive rate
    (FPR= 1- specificity=Type I error) as a threshold for the posterior
    probability of positive class changes from 0 to 1. The point on the
    ROC curve closest to the point (0,1) corresponds to the best
    classifier.
  \item
    AUC (area under the ROC): Overall performance of a classifier
    summarized over all thresholds. AUC measures the probability a
    random positive example is ranked higher than a random negative
    example. A larger AUC indicates a better classifier.
  \end{itemize}
\item
  class-specific prediction performance

  \begin{itemize}
  \tightlist
  \item
    \[\text{precision} = \frac{TP}{TP+FP}=\frac{TP}{\text{predicted postives}}=1-\text{false discovery proportion}\]
  \end{itemize}
\end{itemize}

\section{Homework:}\label{homework}

\begin{itemize}
\tightlist
\item
  Conceptual: 1,2,3,4, 5,6,7,8, 9, 10, 12
\item
  Applied: 13, 14*,15*,16*
\end{itemize}

\section{Code Gist}\label{code-gist-2}

\subsection{Python}\label{python-2}

\subsection{Numpy}\label{numpy-2}

\begin{verbatim}
np.where(lda_prob[:,1] >= 0.5, 'Up','Down')
np.argmax(lda_prob, 1) #argmax along axis=1 (col)
np.asarray(feature_std) # convert to np array
np.allclose(M_lm.fittedvalues, M2_lm.fittedvalues) 
#check if corresponding elts are equal within rtol=1e-5 and atol=-1e08
\end{verbatim}

\subsection{Pandas}\label{pandas-2}

\begin{verbatim}
Smarket.corr(numeric_only=True)
train = (Smarket.Year < 2005)
Smarket_train = Smarket.loc[train] # equivalent to Smarket[train]
Purchase.value_counts() # frequency table
feature_std.std() #calculate column std
S2.index.str.contains('mnth')
Bike['mnth'].dtype.categories # get the categories of the categorical data
obj2 = obj.reindex(["a", "b", "c", "d", "e"])# rearrange the entries in obj according to the new index, introducing missing values if any index values were not already present. 
\end{verbatim}

\subsection{Graphics}\label{graphics-2}

\begin{verbatim}
ax_month.set_xticks(x_month) # set_xticks at the place given by x_month
ax_month.set_xticklabels([l[5] for l in coef_month.index], fontsize=20)
ax.axline([0,0], c='black', linewidth=3,  
          linestyle='--', slope=1);#axline method draw a line passing a given point with a given slope. 
\end{verbatim}

\subsection{ISLP and Statsmodels}\label{islp-and-statsmodels}

\begin{verbatim}
from ISLP import confusion_table
from ISLP.models import contrast

# Logistic Regression using sm.GLM() syntax similar to sm.OLS()
design = MS(allvars)
X = design.fit_transform(Smarket)
y = Smarket.Direction == 'Up'
glm = sm.GLM(y,
             X,
             family=sm.families.Binomial())
results = glm.fit()
summarize(results)
results.pvalues
probs = results.predict() #without data set, calculate predictions on the training set. 
results.predict(exog=X_test) # on test set
# Prediction on a new dataset
newdata = pd.DataFrame({'Lag1':[1.2, 1.5],
                        'Lag2':[1.1, -0.8]});
newX = model.transform(newdata)
results.predict(newX)
confusion_table(labels, Smarket.Direction) #(predicted_labels, true_labels)
np.mean(labels == Smarket.Direction) # calculate the accuracy

hr_encode = contrast('hr', 'sum') #coding scheme for categorical data: the unreported coefficient for the missing level equals to the negative ofthe sum of the coefficients of all other variables. In this a coefficient for a level may be interpreted as the differnece from the mean level of response. 

#Poisson Regression 
M_pois = sm.GLM(Y, X2, family=sm.families.Poisson()).fit()
#`family=sm.families.Gamma()` fits a Gamma regression
model.
\end{verbatim}

\subsection{sklearn}\label{sklearn}

\begin{verbatim}
from sklearn.discriminant_analysis import \
     (LinearDiscriminantAnalysis as LDA,
      QuadraticDiscriminantAnalysis as QDA)
from sklearn.naive_bayes import GaussianNB
from sklearn.neighbors import KNeighborsClassifier
from sklearn.preprocessing import StandardScaler
from sklearn.model_selection import train_test_split
from sklearn.linear_model import LogisticRegression

#LDA
lda = LDA(store_covariance=True) #store the covariance of each class
X_train, X_test = [M.drop(columns=['intercept']) # drop the intercept column
                   for M in [X_train, X_test]]
lda.fit(X_train, L_train) # LDA() model will automatically add a intercept term

lda.means_ # mu_k (n_classes, n_features)
lda.classes_
lda.priors_ # prior probability of each class
#Linear discrimnant vectors
lda.scalings_ #Scaling of the features in the space spanned by the class centroids. Only available for ‘svd’ and ‘eigen’ solvers.

lda_pred = lda.predict(X_test) #predict class labels
lda_prob = lda.predict_proba(X_test) #ndarray of shape (n_samples, n_classes)

#QDA
qda = QDA(store_covariance=True)
qda.fit(X_train, L_train)
qda.covariance_[0] #estimated covariance for the first class

# Naive Bayes
NB = GaussianNB()
NB.fit(X_train, L_train)
NB.class_prior_
NB.theta_ #means for (#classes, #features)
NB.var_ #variances (#classes, #features)
NB.predict_proba(X_test)[:5]

# KNN
knn1 = KNeighborsClassifier(n_neighbors=1)
X_train, X_test = [np.asarray(X) for X in [X_train, X_test]]
knn1.fit(X_train, L_train)
knn1_pred = knn1.predict(X_test)

# When using KNN one should standarize each varaibles
scaler = StandardScaler(with_mean=True,
                        with_std=True,
                        copy=True) # do calculaton on a copy of the dataset
scaler.fit(feature_df)

#train test split
X_std = scaler.transform(feature_df)
(X_train,
 X_test,
 y_train,
 y_test) = train_test_split(np.asarray(feature_std),
                            Purchase,
                            test_size=1000,
                            random_state=0)

# Logistic Regression
logit = LogisticRegression(C=1e10, solver='liblinear') #use solver='liblinear'to avoid warning that the alg doesnot converge.  
logit.fit(X_train, y_train)
logit_pred = logit.predict_proba(X_test)

\end{verbatim}

\subsection{Useful code snippet}\label{useful-code-snippet}

\begin{verbatim}
# Tuning KNN
for K in range(1,6):
    knn = KNeighborsClassifier(n_neighbors=K)
    knn_pred = knn.fit(X_train, y_train).predict(X_test)
    C = confusion_table(knn_pred, y_test)
    templ = ('K={0:d}: # predicted to rent: {1:>2},' +  # > for right alighment
            '  # who did rent {2:d}, accuracy {3:.1%}')
    pred = C.loc['Yes'].sum()
    did_rent = C.loc['Yes','Yes']
    print(templ.format(
          K,
          pred,
          did_rent,
          did_rent / pred))
\end{verbatim}

\bookmarksetup{startatroot}

\chapter{Chapter 5: Resampling
Methods}\label{chapter-5-resampling-methods}

Resampling methods are mainly used to estimate the test error by
resampling the training set. Two methods: cross-validation and
bootstrap. Theses methods refit a model to samples from the training
set, in order to obtain additional information (eg. prediction error on
the test set, standard deviation and bias of estimated parameters) about
the fitted model.

Recall training error in general \emph{dramatically underestimate} the
test error, and in general, training error decreases as the model
flexibility increases, but the test error shows a characteristic U-curve
due to the bias-variance trade-off of the test error.

\emph{Model Assessment}: evaluating a model's performance \emph{Model
selection}: selecting the proper level of flexibility.

\section{how to estimate test error}\label{how-to-estimate-test-error}

\begin{itemize}
\tightlist
\item
  use a large designated test set, but often not available.
\item
  make adjustment to the training error to estimate the test error,
  e.g., Cp statistic, AIC and BIC.
\item
  validation set approach: estimate the test error by \emph{holding out}
  a subset of the training set, also called a \emph{validation set}.

  \begin{itemize}
  \tightlist
  \item
    the estimate of the test error can be highly variable, depending on
    the random train-validation split.
  \item
    Only a subset of the training set is used to fit the model. Since
    statistical methods tend to perform worse when trained on a smaler
    data set, which suggests the validation error tends to
    \emph{overestimate} the test error compared to the model that uses
    the entire training set.
  \end{itemize}
\item
  K-fold Cross-Validation: randomly divide the data into \(K\)
  equal-sized parts \(C_1, C_2, \cdots, C_K\). For each \(k\), leave out
  part \(k\), fit the model on the remaining \(K-1\) parts (combined,
  \((K-1)/K\) of the original traning set), and then evaluate the model
  on the part \(k\). Then repeat this for each \(k\), and weighted
  average of the errors is computed: \[
  CV_{(K)} = \sum_{k=1}^K \frac{n_k}{n}\text{MSE}_k
  \] where \(\text{MSE}_k=\sum_{i\in C_k}(y_i\ne \hat{y}_i)/n_k\).
\end{itemize}

For classification problem, simply replace \(\text{MSE}_k\) with the
misclassificaiton rate
\(\text{Err}_k =\sum_{i\in C_k}I(y_i\ne \hat{y}_i)/n_k\).

The estimated standard error of \(CV_k\) can be calculated by \[
\hat{\text{SE}}(CV_k)=\sqrt{\frac{1}{K}\sum_{k=1}^K\frac{(\text{Err}_k-\overline{\text{Err}_k})^2}{K-1}}
\]

The estimated error tends bias upward because it uses only \((K-1)/K\)
of the training set. This \emph{bias is minimized with \(K=n\) (LOOCV)},
but LOOCV estimate has \emph{high variance} due to the high correlation
between folds.

\begin{itemize}
\item
  LOOCV: it's a special case of K-fold CV with \(K=n\). For least
  squares linear or polynomial regression, the LOOCV error can be
  computed by \[
  \text{CV}_{(n)}=\frac{1}{n}\sum_{i=1}^n \left(\frac{y_i-\hat{y}_i}{1-h_i} \right)^2
  \] Where \(h_i\) is the leverage statistic of \(x_i\). There is no
  randomness in the error. The leverage \(1/n\le h_i\le 1\), reflects
  the amount an observation influences its own fit. The above formula
  doesn't hold in genearl, in which case the model has to refit \(n\)
  times to estimate the test error.
\item
  for LOOCV, the estimate from each fold are highly correlated, hence
  their average can have high variance.
\item
  better choice is \(K=5\) or \(K=10\) for bias-variance trade-off,
  because large \(k\) leads to low bias but high variance due to the
  increased correlation between models. Despite the estimated test error
  sometimes \emph{underestimate} the true test error, they then to be
  close to identify the correct flexibility where the test error is
  minimum.
\item
  Bootstrap: Primarily used to estimate the standard error, or a CI
  (called \emph{bootstrap percentile}) of an estimate . Repeatedly
  sampling the training set with replacement and obtain a
  \emph{bootstrap set} of the \emph{the same size} as the original
  training set. One can fit a model and estimate a parameter with each
  bootstrap data set, and then estimate the \emph{standard error} using
  the estimated parameters by the bootstrap model, assuming there are
  \(B\) bootstrap data sets: \[
  SE_B(\hat{\alpha})=\sqrt{\frac{1}{B-1}\sum_{r=1}^B (\hat{\alpha}^{*r}-\bar{\hat{\alpha}}^*)^2   }
  \]

  \begin{itemize}
  \item
    Note sometimes sampling with replacement must take caution, for
    example, one can't simply sample a time series with replacement
    because the data are sequential.
  \item
    Estimate prediction error: Each bootstrap sample has significant
    overlap with the original data, in fact, about 2/3 of the original
    data points appear in each bootstrap sample. If we use the original
    data set as the validation set, This will cause the bootstrap to
    seriously \emph{underestimate} the true prediction error. To fix
    this, one can only use predictions on those samples that do not
    occur (by chance) in a bootstrap sample.
  \item
    Bootstrap vs.~Permutation test: permutation methods sample from an
    estimated \emph{null} distribution for the data, and use this to
    estimate \(p\)-values and \emph{False Discovery Rates} for
    hypothesis tests.

    The bootstrap can be used to test a null hypothesis in simple
    situation. Eg. If \(H_0: \theta=0\), we can check whether the
    confidence interval for \(\theta\) contains zero.
  \end{itemize}
\end{itemize}

\section{Homework}\label{homework-1}

\begin{itemize}
\tightlist
\item
  Conceptual: 1,2,3,4
\item
  Applied: 5--9, at least one.
\end{itemize}

\section{Code Gist}\label{code-gist-3}

\subsection{Python}\label{python-3}

\begin{verbatim}
np.empty(1000) #create an array without initializing
quartiles = np.percentile(arr, [25, 50, 75])
\end{verbatim}

\subsection{Numpy}\label{numpy-3}

\begin{verbatim}
c = np.power.outer(row, col) # mesh of row[i]^col[j] power. 
# random choice 
rng = np.random.default_rng(0)
alpha_func(Portfolio,
           rng.choice(100, # random numbers are selected from arange(100)
                      100, #size
                      replace=True))

    
\end{verbatim}

\subsection{Pandas}\label{pandas-3}

\begin{verbatim}
np.cov(D[['X','Y']].loc[idx], rowvar=False) #cov compute corr of variables. rowvar-False: cols are vars.
\end{verbatim}

\subsection{Graphics}\label{graphics-3}

\subsection{ISLP and statsmodels}\label{islp-and-statsmodels-1}

\begin{verbatim}
# function that evalues MSE for training a model
def evalMSE(terms,    #predictor variables
            response, #response variable
            train,
            test):

   mm = MS(terms)
   X_train = mm.fit_transform(train)
   y_train = train[response]

   X_test = mm.transform(test)
   y_test = test[response]

   results = sm.OLS(y_train, X_train).fit()
   test_pred = results.predict(X_test)

   return np.mean((y_test - test_pred)**2)

# Compare polynomial models of different degrees
MSE = np.zeros(3)
for idx, degree in enumerate(range(1, 4)):
    MSE[idx] = evalMSE([poly('horsepower', degree)],
                       'mpg',
                       Auto_train,
                       Auto_valid)
MSE

# Estimating the accuracy of a LR model using bootstrap

# Compute the SE of the boostraped values computed by func                      
def boot_SE(func,
            D,
            n=None,
            B=1000,
            seed=0):
    rng = np.random.default_rng(seed)
    first_, second_ = 0, 0
    n = n or D.shape[0]
    for _ in range(B):
        idx = rng.choice(D.index,
                         n,
                         replace=True)
        value = func(D, idx)
        first_ += value
        second_ += value**2
    return np.sqrt(second_ / B - (first_ / B)**2) #compute var. 
def boot_OLS(model_matrix, response, D, idx):
    D_ = D.loc[idx]
    Y_ = D_[response]
    X_ = clone(model_matrix).fit_transform(D_) #clone create a deep copy. 
    return sm.OLS(Y_, X_).fit().params
    
quad_model = MS([poly('horsepower', 2, raw=True)]) #raw=True: not normalize the feature
quad_func = partial(boot_OLS,
                    quad_model,
                    'mpg')
boot_SE(quad_func, Auto, B=1000)
\end{verbatim}

\subsection{sklearn}\label{sklearn-1}

\begin{verbatim}
from functools import partial
from sklearn.model_selection import \
     (cross_validate,
      KFold,
      ShuffleSplit)
from sklearn.base import clone
from ISLP.models import sklearn_sm #wrapper to feed a sm model to sklearn

#Cross Validation
hp_model = sklearn_sm(sm.OLS,
                      MS(['horsepower']))
X, Y = Auto.drop(columns=['mpg']), Auto['mpg']
cv_results = cross_validate(hp_model,
                            X,
                            Y,
                            cv=Auto.shape[0]) #cv=K.loocv. Can use cv=KFold()object
cv_err = np.mean(cv_results['test_score']) # test_score: MSE
cv_err

# Use KFold to partition instead of using an integer. 
cv_error = np.zeros(5)
cv = KFold(n_splits=10,
           shuffle=True,#shuffle before splitting
           random_state=0) # use same splits for each degree
for i, d in enumerate(range(1,6)):
    X = np.power.outer(H, np.arange(d+1))
    M_CV = cross_validate(M,
                          X,
                          Y,
                          cv=cv)
    cv_error[i] = np.mean(M_CV['test_score'])
cv_error

# using ShuffleSplit() method 
validation = ShuffleSplit(n_splits=10,
                          test_size=196,
                          random_state=0)
results = cross_validate(hp_model,
                         Auto.drop(['mpg'], axis=1),
                         Auto['mpg'],
                         cv=validation)
results['test_score'].mean(), results['test_score'].std()


#View skleanrn fitting results using model.results_
hp_model.fit(Auto, Auto['mpg']) # hp_model is a sklearn model sk_model.fit(X, Y) for trainning
model_se = summarize(hp_model.results_)['std err'] #summarize is an ISLP function
model_se
\end{verbatim}

\subsection{Useful code snippet}\label{useful-code-snippet-1}

\begin{verbatim}

\end{verbatim}

\bookmarksetup{startatroot}

\chapter{Chapter 6: Linear Model Selection and
Regrularization}\label{chapter-6-linear-model-selection-and-regrularization}

Linear models are \emph{interpretable} and often shows small variance.
They are fitted by \emph{OLS}. There are other methods that can either
provide alternatives or improve linear regression models in terms of
\emph{prediction accuracy} (especially when \(p>n\)) and automatic
\emph{feature selection} for improved interpretability.

There are three classes of methods:

\begin{itemize}
\tightlist
\item
  subset selection: pick a subset of the \(p\) predictors that best
  explains the response.
\item
  Shrinkage (regularization). With an added regularizing term, estimated
  parameters are shrunken to zero relative the OLS estimates. If \(L^2\)
  regularization is used, all coefficients are shrunk toward zero by the
  same proportion; while if \(L^1\) is used, then some coefficients will
  become zero, leading to actual \emph{variable selection} or
  \emph{sparse representation}.
\item
  Dimension reduction. Project \(p\)-predictors to a \(M\) (\(M<p\))
  dimensional subspace. Each new direction is a linear combination (or
  projection) of the \(p\)-variables. These \(M\)-projections can then
  be used to fit a linear regression model(\textbf{PCR} if the
  \(M\)-projections are obtained in an unsupervised way; or \textbf{PLR}
  if these \(M\)-projections are obtained in a supervised way))
\end{itemize}

\section{Best Subset Selecttion}\label{best-subset-selecttion}

\textbf{Algorithm}

\begin{enumerate}
\def\labelenumi{\arabic{enumi}.}
\tightlist
\item
  Fit the data with the \emph{null model} \(\mathcal{M}_0\), which
  contains no predictors. This model simply set \(Y=\text{mean}(y_i)\).
\item
  for \(k=1, 2, \cdots, p\): fit \(p \choose k\) models containing
  exactly \(k\) predictors. Pick the best one that having the smallest
  RSS or largest \(R^2\) (or \emph{deviance} for classification problem,
  i.e., \(-2\max \log (\text{likelihood})\) on the \emph{training set},
  called \(\mathcal{M}_k\). Note for each categorical variable with
  \(L\)-level, there are \(L-1\) dummy variables.
\item
  Select the best one among \(\mathcal{M}_0, \cdots, \mathcal{M}_p\)
  using cross-validation or other measures such as \(C_p (AIC)\),
  \(BIC\) or adjusted \(R^2\). If cross-validation is used, then Step 2
  is repeated on each training fold, and the validation errors are
  averaged to select the best \(k\). The the model \(\mathcal{M}_k\) fit
  on the full training set is delivered for chosen \(k\).
\end{enumerate}

Best subset selection suffers - high computation: needs to compute
\(2^p\) models - overfitting due to the large search space of models

\section{Stepwise selection}\label{stepwise-selection}

Both Forward and Backward selection are stepwise selection. They are
used when \(p\) is large. They searches over \(1+p(p+1)/2\) models and
are \emph{greedy} algorithm and is not guaranteed to find the best
possible model out of all \(2^p\) models.

\begin{itemize}
\tightlist
\item
  Backward selection requires \(n>p\) (so that the full model can be
  fit);
\item
  Forward selection can be used when \(n<p\) but only fits up to models
  with \(n\) variables.
\item
  One can combine forward and backward selection to a hybrid selection.
\end{itemize}

\subsection{Forward Stepwise
Selection}\label{forward-stepwise-selection}

Adding one variable at at time that offers the greatest additional
improvement.

\textbf{Algorithm}

\begin{enumerate}
\def\labelenumi{\arabic{enumi}.}
\tightlist
\item
  Fit the data with the \emph{null model} \(\mathcal{M}_0\), which
  contains no predictors. This model simply set \(Y=\text{mean}(y_i)\).
\item
  for \(k=1, 2, \cdots, p-1\):
\end{enumerate}

\begin{itemize}
\tightlist
\item
  fit all \(p-k\) models that augment the predictors in
  \(\mathcal{M}_k\) with one additional predictor.
\item
  Pick the best one that having the smallest RSS or largest \(R^2\) on
  the training set, called \(\mathcal{M}_{k+1}\).
\end{itemize}

\begin{enumerate}
\def\labelenumi{\arabic{enumi}.}
\setcounter{enumi}{2}
\tightlist
\item
  Select the best one among \(\mathcal{M}_0, \cdots, \mathcal{M}_p\)
  using cross-validation or other measures such as \(C_p (AIC)\),
  \(BIC\) or adjusted \(R^2\).
\end{enumerate}

\subsection{Backward Stepwise
Selection}\label{backward-stepwise-selection}

It begins with the full model with all variables, and iteratively
removing one variable at at time.

\textbf{Algorithm}

\begin{enumerate}
\def\labelenumi{\arabic{enumi}.}
\tightlist
\item
  Fit the data with the \emph{full model} \(\mathcal{M}_p\), which
  contains all predictors.
\item
  for \(k=p, p-1, \cdots, 1\):
\end{enumerate}

\begin{itemize}
\tightlist
\item
  fit all \(k\) models that contains all but one of the predictors in
  \(\mathcal{M}_k\).
\item
  Pick the best one that having the smallest RSS or largest \(R^2\) on
  the training set, called \(\mathcal{M}_{k-1}\).
\end{itemize}

\begin{enumerate}
\def\labelenumi{\arabic{enumi}.}
\setcounter{enumi}{2}
\tightlist
\item
  Select the best one among \(\mathcal{M}_0, \cdots, \mathcal{M}_p\)
  using cross-validation or other measures such as \(C_p (AIC)\),
  \(BIC\) or adjusted \(R^2\).
\end{enumerate}

\section{Model selection}\label{model-selection-1}

Models with all predictors always have the smallest \(RSS\) or largest
\(R^2\) on the \emph{training set}. Therefore they are not suitable to
choose the best one among models with different number of predictors.

We ought to \emph{estimate the test error} on a test set. This may be
done - indirectly by adjusting the training error to account for the
bias due to overfitting: \(C_p\) (equivalently, AIC in case of linear
model with Gaussian errors), BIC and adjusted \(R^2\).

\begin{itemize}
\item
  Mallow's \(C_p\): \[
    C_p =\frac{1}{n}(RSS+2d\hat{\sigma}^2)
    \] where, \(d\) us the number of parameters and
  \(\hat{\sigma}^2 \approx Var{\epsilon}\), typically estimated by using
  the \emph{full model} containing all variables. \(C_p\) is an unbiased
  estimate of test MSE.
\item
  AIC is defined for models fit by maximmum likelihood. \[
    AIC = -2\log L + 2d
    \] where, \(L\) is the maximum likelihood function for the estimated
  model. For linear regression with Gaussian error, \(AIC\propto C_p\).
\item
  BIC \[
    BIC = \frac{1}{n}( RSS + \log (n)d \hat{\sigma}^2 )
    \] Since \(\log n>\) for \(n>7\), the BIC places a higher penalty on
  models with many variables, and hence select smaller models than
  \(C_p\).
\item
  Adjusted \(R^2\) (larger value is better)

  \[
  \text{Adjusted }R^2=1-\frac{RSS/(n-d-1)}{TSS/(n-1)}
  \] \(RSS/(n-d-1)\) may increase or decrease depends on \(d\). Unlike
  \(R^2\), adjusted \(R^2\) pays a price for the inclusion of
  unnecessary variables in a model.

  \(C_p\), AIC, BIC, adjusted \(R^2\),are not appropriate in
  high-dimentional setting, as the estimated
  \(\hat{\sigma}^2 \approx 0\) (when \(p\ge n\)).
\item
  directly by cross-validation (or validation). It does not require
  estimate \(\sigma^2\). It has a wide range of usage, as it may
  difficult to estimate \(d\) or \(\sigma^2\). One can choose the model
  that has the smallest test error or using the \emph{one-standard-error
  rule} to select the model that has a smaller size:

  \begin{itemize}
  \tightlist
  \item
    calculate the SE of the estimated test MSE for each model size.
  \item
    identify the lowest test MSE
  \item
    choose the smallest model for which its test MSE is within one SE of
    the lowest point.
  \end{itemize}
\end{itemize}

\section{Shrinkage methods for Variable
selection}\label{shrinkage-methods-for-variable-selection}

The shrinkage offers an alternative to selecting variables by adjusting
a hyperparameter that trades-off RSS and the model parameter magnitudes.
Shrinkage methods will produce a different set of coefficients for a
different \(\lambda\). Cross-validation may be used to select the best
\(\lambda\). After the \(\lambda\) is selected, one can fit a final
model using the entire training data set.

The reason shrinkage methods may perform better than OLS is rooted in
bias-variance trade-off: as \(\lambda\) increases, the flexibility of
the model decreases b ecause of shrunk coefficients, leading to
decreased variance but increased bias.

\subsection{Ridge regression: minimize the following
objective}\label{ridge-regression-minimize-the-following-objective}

\[
RSS + \lambda \sum_{j=1}^p \beta_j^2
\] The ridge regression is equivalent to \[
\text{minimize } RSS \qquad \text{subject to } \sum_{j=1}^p \beta_j^2 \le s
\] for some \(s\ge 0\).

\begin{itemize}
\item
  It encourages the model parameters to shrink toward zero and find a
  balance between RSS and model parameter magnitudes. Cross-validation
  is used to find the best tuning parameter \(\lambda\). When
  \(\lambda\) is large, \(\beta_j\to 0\). Note, Ridge shrinks all
  coefficients and include all \(p\) variables.
\item
  The OLS coefficients estimates are \emph{scale equivariant}:
  regardless of how \(X_j\) is scaled, \(X_j\hat{\beta}_j\) remain the
  same: if \(X_j\) is multiplied by \(c\), this will simply leads to
  \(\hat{\beta}_j\) be scaled by a factor of \(1/c\).
\item
  In contrast, when multiplying \(X_j\) by a factor, this may
  significantly change the ridge coefficients. Ridge coefficients
  depends on \(\lambda\) and the scale of \(X_j\), and may even on the
  scaling of other predictors. Therefore, it is best practice to
  \emph{standardize the predictors} before fitting a ridge model: \[
  \tilde{x}_{ij} =\frac{x_{ij}}{\frac{1}{n} \sum_{i=1}^n(x_{ij} - \bar{x}_j)}
  \]
\item
  Ridge regression works best in situations where the OLS estimates have
  high variance, especially when \(p\) is large.
\item
  Ridge will include all \(p\) variables in the final model.
\end{itemize}

\subsection{The Lasso (Least Absolute Shrinkage and Selection
Operator)}\label{the-lasso-least-absolute-shrinkage-and-selection-operator}

\begin{itemize}
\item
  The Lasso replaces the \(\ell^2\) error with \(\ell^1\) penalty. Lasso
  can force some coefficients to become exactly zero when \(\lambda\) is
  large enough. Thus it can actually performs \emph{variable selection}
  hence better interpretation. Again, cross-validation is employed to
  select \(\lambda\).
\item
  The reason Lasso can perform variable selection is because the
  objective function is equivalent to
\end{itemize}

\[\text {minimizing RSS}, \text{subject to } \sum_{j=1}^p |\beta_j| \le s
  \] for some \(s\). The contour of RSS in general only touch the
\(\ell_1\) ball at its vertex, at which a minimum is obtained with some
variables vanishes. In contrast, in the ridge situation, the \(\ell_2\)
ball is round, and in general, the contour of the RSS function only
touches the sphere at a surface point where a minimum is obtained with
no variable vanishes.

\begin{itemize}
\item
  Neither ridge nor the lasso will universally dominate the other. When
  the response depends on a small number of predictors, one may expect
  lasso performs better; but in practice, this is never known in
  advance.
\item
  Combining ridge and lasso leads to \emph{elastic net} method.
\item
  it is well known that ridge tends to give similar values coefficient
  values to correlated variables, while lasso may give quite different
  coefficient values to correlated variables.
\item
  ridge regression shrinks all coefficients by the same proportion.
  While lasso perform \textbf{soft-thresholding}, shrink all
  coefficients by similar amount, and sufficient small coefficients are
  shrunk all the way to zero.

  both ridge and lasso can be considered as computationally feasible
  approximation to the \emph{best subset selection} which can be
  equivalently formulated as: \[
  \text{minimize } RSS \qquad \text{subject to } \sum_{j=1}^p I(\beta_j\ne 0) \le s.
  \]
\item
  \textbf{Bayesian formulation}: Both ridge and lasso can be interpreted
  as maximize the \textbf{posterior probability} (MAP) \[
  p(\beta|X,Y)\propto f(Y|X,\beta) p(\beta|X)=f(Y|X,\beta)p(\beta)
  \] where \(p(\beta)= \prod_{j=1}^p g(\beta_j)\) with some density
  function \(g\) is the believed prior on \(\beta\).

  \begin{itemize}
  \tightlist
  \item
    if \(g\) is Gaussian with mean zero and standard deviation a
    function of \(\lambda\), then it follows the solution \(\beta\)
    given by the ridge is the same as maximizing the posterior
    \(p(\beta|X,Y)\), that is, \(\beta\) is the posterior mode. In fact,
    \(\beta\) is also the posterior mean. Since the Gaussian prior is
    flat at near zero, ridge assumes the coefficients are randomly
    distributed about zero.
  \item
    if \(g\) is double-exponential (Laplace) with mean zero and scale
    parameter a function of \(\lambda\), then it follows the solution
    \(\beta\) given by the lasso is the same as maximizing the posterior
    \(p(\beta|X,Y)\), that is, \(\beta\) is the posterior mode. In this
    case \(\beta\) is \textbf{not} the posterior mean. Since the
    Laplacian prior is steeply peaked at zero, lasso expects a priori
    that many coefficients are (exactly) zero.
  \end{itemize}
\end{itemize}

\section{\texorpdfstring{Dimension reduction methods: transforming
\(X_j\).}{Dimension reduction methods: transforming X\_j.}}\label{dimension-reduction-methods-transforming-x_j.}

There are two types of dimension reduction methods for regression: a)
PCA regression, b) Partial list squares PLS. Both are designed to handle
when the OLS breaks down due to that there are large number of
correlated variables.

\subsection{\texorpdfstring{PCA regression: first use PCA to obtain
\(M\)- PCA as linear combinations (directions) of the original \(p\)
predictors:}{PCA regression: first use PCA to obtain M- PCA as linear combinations (directions) of the original p predictors:}}\label{pca-regression-first-use-pca-to-obtain-m--pca-as-linear-combinations-directions-of-the-original-p-predictors}

\begin{equation}\phantomsection\label{eq-PCA}{
  Z_m =\sum_{j=1}^p \phi_{mj} X_j, \qquad 1\le m \le M,  
  }\end{equation} where, the \(\phi_{mj}\) are called \textbf{PCA
loadings}, and subject to the norm \(\sum_{j=1}^p\phi_{mj}^2=1\) for
each \(m\). Note \(Z_m\) is a vector of length equal to the length of
\(X_j\), which is the number of data points \(n\). The component of
\(Z_m\): \(z_{im}\), \(1\le i \le n\) are called \textbf{PCA scores}.
\(z_{im}\) is a \emph{single number summary} of the \(p\) predictors
with the \(m\)-th PCA for the \(i\)-th observation. PCA is not a feature
selection method.

The first PCA defines the direction that contains the largest variance
in \(X\), and minimize the sum of squared perpendicular distances to
each point (the projection error on the PCA), that is it defines the
line that is \emph{as close as possible} to the data; (In fact, the
first PCA is given by the eigenvector of the largest eigenvalue of the
covariance matrix \(\frac{1}{n-1}X^TX\)). The second PCA is orthogonal
to the first PCA and has the second largest variance and is uncorrelated
with the first, and so on. These directions are obtained in an
\emph{unsupervised way}, as \(Y\) is not used to obtain these
components. Consequently, there is no guarantee that the directions that
best explain the predictors will also be the best directions to use for
predicting the response.

PCA is typically conducted after standardizing the data \(X\), as
without scaling, the high variance variables will tend to have higher
influence on the obtained PCAs.

We then use OLS to fit a linear regression model
\begin{equation}\phantomsection\label{eq-PCAR}{
y_i =\theta_0 +\sum_{m=1}^M \theta_m z_{im}+\epsilon_i, \qquad i=1,2,\cdots, n
}\end{equation}

After substitute Equation~\ref{eq-PCA} into equation
Equation~\ref{eq-PCAR}, one can find that \[
\sum_{m=1}^M \theta_mz_{im} =\sum_{j=1}^p \beta_jx_{ij}
\] with \begin{equation}\phantomsection\label{eq-pcar-beta}{
\beta_j = \sum_{m=1}^M \theta_m\phi_{mj}.
}\end{equation} Eq. Equation~\ref{eq-pcar-beta} has the potential to
bias the coefficient estimates, but selecting \(M<< p\) can
significantly reduce the variance. So model Equation~\ref{eq-PCAR} is a
special case of linear regression subject to the constants
Equation~\ref{eq-pcar-beta}.

PCR and ridge are closely related and one can think of ridge regression
as a continuous version of PCR.

\subsection{Partial Least Squares}\label{partial-least-squares}

Similar to PCAR, PLS also first identifies a new set of features
\(Z_1, Z_2, \cdots, Z_m\), each of which is a linear combinations of the
original features, and then fits a linear model via OLS with these new
\(M\) features.

But PLS identifies these new features in a \emph{supervised way}, that
is, PLS uses \(Y\) in order to identify the new features that not only
approximate the old features well, but also are \emph{related to the
response}, i.e., these new features explain both the response and the
predictors.

First PLS standardizes the \(p\) predictors. PLS identifies the first
component \(Z_1 = \sum_{j=1}^p \phi_{1j}X_j\) by choosing
\(\phi_{1j}=<X_j, Y>\), the coefficient from the simple linear
regression of \(Y\) onto \(X_j\). Since this coefficient is equal to the
correlation between \(Y\) and \(X_j\), PLS places the highest weight on
the variables that are most strongly related to \(Y\). The PLS direction
does not fit the predictors as closely as does PCA, but it does a better
job explaining the response.

Next, PLS orthogonality each \(X_j\) with respect to \(Z_1\), that is,
replace each \(X_j\) with the residual by regressing \(X_j\) on \(Z_1\),
and then repeat the same process.

When \(p\) is large, especially \(p>n\), the forward selection method,
shrinkage methods (lasso or ridge), PCR, PLR fit a \emph{less flexible}
model, hence particularly useful in performing regression in
high-dimensional settings.

\section{Homework:}\label{homework-2}

\begin{itemize}
\tightlist
\item
  Conceptual: 1--4
\item
  Applied: At least one.
\end{itemize}

\section{Code Snippet}\label{code-snippet}

\subsection{Python}\label{python-4}

\begin{verbatim}
np.isnan(Hitters['Salary']).sum()
\end{verbatim}

\subsection{Numpy}\label{numpy-4}

\begin{verbatim}
np.linalg.norm(beta_hat) #L2 norm. ord=1: L1  ord='inf': max norm.
\end{verbatim}

\subsection{Pandas}\label{pandas-4}

\begin{verbatim}
Hitters.dropna();
soln_path = pd.DataFrame(soln_array.T,
                         columns=D.columns,
                         index=-np.log(lambdas))
soln_path.index.name = 'negative log(lambda)'
\end{verbatim}

\subsection{Graphics}\label{graphics-4}

\begin{verbatim}
ax.errorbar(np.arange(n_steps), 
            cv_mse.mean(1), #mean of each row (model)
            cv_mse.std(1) / np.sqrt(K), #estimate standard error of the mean
            label='Cross-validated',
            c='r') # color red
            
ax.axvline(-np.log(tuned_ridge.alpha_), c='k', ls='--') # plot a verticalline
\end{verbatim}

\subsection{ISLP and statsmodels}\label{islp-and-statsmodels-2}

\begin{verbatim}
#Estimate Var(epsilon)

design = MS(Hitters.columns.drop('Salary')).fit(Hitters)
design.terms # to see the variable names in the design matrix
Y = np.array(Hitters['Salary'])
X = design.transform(Hitters)
sigma2 = OLS(Y,X).fit().scale  #.scale: RSE: residual standard error estimating 

# Forward Selection using ISLP.models and a scoring function
from ISLP.models import \
     (Stepwise,
      sklearn_selected,
      sklearn_selection_path)
strategy = Stepwise.first_peak(design,
                               direction='forward',
                               max_terms=len(design.terms))
hitters_Cp = sklearn_selected(OLS,
                               strategy,
                               scoring=neg_Cp)
                               #default scoring MSE, will choose all variables
hitters_Cp.fit(Hitters, Y) # the same as hitters_Cp.fit(Hitters.drop('Salary', axis=1), Y)
hitters_Cp.selected_state_

#Forward selection using cross-validation
strategy = Stepwise.fixed_steps(design,
                                len(design.terms),
                                direction='forward')
full_path = sklearn_selection_path(OLS, strategy) #using default scoring MSE
full_path.fit(Hitters, Y) # there are , 19 variables, 20 models
Yhat_in = full_path.predict(Hitters)

#calculate in-sample mse

mse_fig, ax = subplots(figsize=(8,8))
insample_mse = ((Yhat_in - Y[:,None])**2).mean(0) #Y[:,None]: add a second axis, create a column vector
                        #[yw] mean(0): calculate mean along row, i.e., for each col. mean(1): calculate mean for each row

#Cross-validation
K = 5
kfold = skm.KFold(K,
                  random_state=0,
                  shuffle=True)
Yhat_cv = skm.cross_val_predict(full_path,
                                Hitters,
                                Y,
                                cv=kfold)
# Cross-validation mse
cv_mse = []
for train_idx, test_idx in kfold.split(Y):
    errors = (Yhat_cv[test_idx] - Y[test_idx,None])**2
    cv_mse.append(errors.mean(0)) # column means
cv_mse = np.array(cv_mse).T

#validation approach using ShuffleSplit
validation = skm.ShuffleSplit(n_splits=1, # only split one time. 
                              test_size=0.2,
                              random_state=0)
for train_idx, test_idx in validation.split(Y):
    full_path.fit(Hitters.iloc[train_idx], #note needing to use iloc
                  Y[train_idx])
    Yhat_val = full_path.predict(Hitters.iloc[test_idx])
    errors = (Yhat_val - Y[test_idx,None])**2
    validation_mse = errors.mean(0)
\end{verbatim}

\subsection{sklearn}\label{sklearn-2}

\begin{verbatim}
rom sklearn.pipeline import Pipeline
from sklearn.decomposition import PCA
from sklearn.cross_decomposition import PLSRegression

#Best subset selection using 10bnb

D = design.fit_transform(Hitters)
D = D.drop('intercept', axis=1) #needs to drop intercept
X = np.asarray(D)
path = fit_path(X, 
                Y,
                max_nonzeros=X.shape[1]) #fit_path: a funciton from l0nb. use all variables
                # max_nonzeros: max nonzero coefficients in the fitted model.

# Ridge Regression
soln_array = skl.ElasticNet.path(Xs, # standardized, no intercept
                                 Y,
                                 l1_ratio=0., #ridge
                                 alphas=lambdas)
# Using pipline
ridge = skl.ElasticNet(alpha=lambdas[59], l1_ratio=0)
scaler = StandardScaler(with_mean=True,  with_std=True)
pipe = Pipeline(steps=[('scaler', scaler), ('ridge', ridge)])
pipe.fit(X, Y)
ridge.coef_

# Validation

validation = skm.ShuffleSplit(n_splits=1,
                              test_size=0.5,
                              random_state=0) # validation is a generator
ridge.alpha = 0.01
results = skm.cross_validate(ridge,
                             X,
                             Y,
                             scoring='neg_mean_squared_error',
                             cv=validation) # using the strategy defined in validation
-results['test_score']

# GridSearchCV()
param_grid = {'ridge__alpha': lambdas}
grid = skm.GridSearchCV(pipe,
                        param_grid,
                        cv=validation, # or use cv=kfold (5-fold CV defined separately)
                        scoring='neg_mean_squared_error') #default scoring=R^2
grid.fit(X, Y)
grid.best_params_['ridge__alpha']
grid.best_estimator_
grid.cv_results_['mean_test_score']
grid.cv_results_['std_test_score']


#Plot CV MSE
ridge_fig, ax = subplots(figsize=(8,8))
ax.errorbar(-np.log(lambdas),
            -grid.cv_results_['mean_test_score'],
            yerr=grid.cv_results_['std_test_score'] / np.sqrt(K))
ax.set_ylim([50000,250000])
ax.set_xlabel('$-\log(\lambda)$', fontsize=20)
ax.set_ylabel('Cross-validated MSE', fontsize=20);

# Use ElasticNetCV()
ridgeCV = skl.ElasticNetCV(alphas=lambdas, # ElasticNetCV accepts a sequence of alphas
                           l1_ratio=0,
                           cv=kfold)
pipeCV = Pipeline(steps=[('scaler', scaler), # scaling is done once. 
                         ('ridge', ridgeCV)])
pipeCV.fit(X, Y)
tuned_ridge = pipeCV.named_steps['ridge']
tuned_ridge.mse_path_
tuned_ridge.alpha_ # best alpha
tuned_ridge.coef_

# Evaluating test Error of Cross-validated Ridge 

outer_valid = skm.ShuffleSplit(n_splits=1, 
                               test_size=0.25,
                               random_state=1)
inner_cv = skm.KFold(n_splits=5,
                     shuffle=True,
                     random_state=2)
ridgeCV = skl.ElasticNetCV(alphas=lambdas, # a sequence of lambdas
                           l1_ratio=0,
                           cv=inner_cv) # K-fold validation
pipeCV = Pipeline(steps=[('scaler', scaler),
                         ('ridge', ridgeCV)]);
                         
                         
results = skm.cross_validate(pipeCV, 
                             X,
                             Y,
                             cv=outer_valid,
                             scoring='neg_mean_squared_error')
-results['test_score']

# Lasso regression
lassoCV = skl.ElasticNetCV(n_alphas=100, #test 100 alpha values
                           l1_ratio=1,
                           cv=kfold)
pipeCV = Pipeline(steps=[('scaler', scaler),
                         ('lasso', lassoCV)])
pipeCV.fit(X, Y)
tuned_lasso = pipeCV.named_steps['lasso']
tuned_lasso.alpha_
tuned_lasso.coef_
np.min(tuned_lasso.mse_path_.mean(1)) # miminum avg mse

#to get the soln path
lambdas, soln_array = skl.Lasso.path(Xs, # standarsized, no -intercept
                                    Y,
                                    l1_ratio=1,
                                    n_alphas=100)[:2]

#PCA and PCR
pca = PCA(n_components=2)
linreg = skl.LinearRegression()
pipe = Pipeline([('scaler', scaler), 
                 ('pca', pca),
                 ('linreg', linreg)])
pipe.fit(X, Y)
pipe.named_steps['linreg'].coef_
pipe.named_steps['pca'].explained_variance_ratio_

# perform Grid search
param_grid = {'pca__n_components': range(1, 20)} #PCA needs n_components >0
grid = skm.GridSearchCV(pipe,
                        param_grid,
                        cv=kfold,
                        scoring='neg_mean_squared_error')
grid.fit(X, Y)

# cross-validation a null model
cv_null = skm.cross_validate(linreg,
                             Xn,
                             Y,
                             cv=kfold,
                             scoring='neg_mean_squared_error')
-cv_null['test_score'].mean()

#PLS
pls = PLSRegression(n_components=2, 
                    scale=True) # standarsize the data 
pls.fit(X, Y) # X has no-intercept 

# Cross-validation
param_grid = {'n_components':range(1, 20)}
grid = skm.GridSearchCV(pls,
                        param_grid,
                        cv=kfold,
                        scoring='neg_mean_squared_error')
grid.fit(X, Y)

\end{verbatim}

\bookmarksetup{startatroot}

\chapter{Chapter 7: Moving Beyond
Linearity}\label{chapter-7-moving-beyond-linearity}

Often the linearity assumption of Y of X is good but it may have limited
predictive power, as often a linear model is just an approximation.
Imporovement can be obtained by reducing the model complexity (hence
variance) with lasso, ridge, PCAR or PLR, etc. when it's not, -
Polynomials - step functions - splines - local regression, - generalized
additive models offer much more flexibility. All the above nonlinear
methods falls into the \emph{basis function} approach where we fit the
following function \[
y_i=\beta_0+\beta_1b_1(x_i) +\cdots + \beta_K b_K(x_i) +\epsilon_i
\] where the \(b_j\), \(j=1, \cdots, K\) are pre-defined basis functions
that transform \(X\) to a feature \(b_j(X)\). Each approach
corresponding to a choice of particular family of basis functions. The
model can then fit with \emph{OLS}.

\section{Polynomials}\label{polynomials}

The basis functions are simply the polynomial functions of different
degrees. Polynomial terms of higher powers for \(X_j\) or interaction
terms \(X_iX_j\) are used, but the model is still a linear model in the
coefficients \(\beta_j\). The optimum degree \(d\) can be chosen by
cross-validation. polynomial terms can be included in either a linear
regression model or a logistic regression model.

In practice hardly a degree greater than 3 or 4 is used because a higher
degree polynomial exhibits high degree of oscillation, especially near
the boundary (Runge's phenomenon). This is because a polynomial imposes
a \emph{global structure}.

The standard error at a point \(x_0\)is calculated by \[
SE[\hat{f}(x_0)] = \ell_0^T \hat{\bf{C}} {\ell}_0
\] where, \(\ell_0^T =(1, x_0, x_0^2, \cdots, x_0^d)\), and
\(\hat{\bf{C}}\) is the covariance matrix of the estimated coefficients
\(\beta_j\), \(j=0, 1, \cdots, d\) obtained from the OLS.

\section{Step functions}\label{step-functions}

A step function is a piece-wise constant function. Cut a \(X\) variable
into \(K+1\) regions using \(K\) cut points and then either use one-hot
coding with \(K+1\) dummy variables (and no intercept, in this case,
each coefficient can be interpreted as the average value in that region)
or create \(K\) dummy variables with an intercept to represent all those
regions (in this case, the average value in that region equals to the
intercept plus the coefficient). Choice of \emph{cut-points} (knots) can
be problematic. Binning the \(X\) variable amounts to convert a
continuous variable into an \emph{ordered categorical variable}.

The basis functions are simply \emph{indicator functions} on each
region: \[
b_j(x_i)= I(c_j\le x_i< c_{j+1}).
\]

\section{Piececwise polynomials}\label{piececwise-polynomials}

It overcomes the disadvantage of polynomial basis which imposes a
\emph{global structure}. It fits separate low-degree polynomials over
different regions of \(X\) separated by \emph{knots}.

\section{Splines}\label{splines}

Splines are piece-wise polynomials of degree \(d\) that are continuous
up to \(d-1\) derivatives at each knot. E.g. a cubic spline with \(K\)
knots has continuity up to second derivative at each knot, and has
degree of freedom of \(K+4\).

\begin{itemize}
\item
  linear spline: with knots \(\xi_k\), \(k=1, \cdots, K\) is a
  piece-wise linear polynomial that is continuous at each knot. \[
  y= \beta_0+\beta_1 b_1(x)+\cdots + \beta_{K+1}b_{K+1}(x)+\epsilon, 
  \] where, \(b_k\) are \textbf{basis functions} defined by
  \begin{align}
  b_1(x) & = x \\
  b_{k+1}(x) & = (x- \xi_k)_{+}, \qquad k=1, \cdots K
  \end{align} Here the \emph{positive part} is defined by
  \[x_{+}=\begin{cases}
  x, &  \text{ if } x>0 \\
  0 &  \text{ otherwise}
  \end{cases}
  \]
\item
  cubic splines: with knots \(\xi_k\), \(k=1, \cdots, K\) is a piecewise
  cubic polynomials with continuous derivatives up to order 2 at each
  knot.
\end{itemize}

\[
y= \beta_0+\beta_1 b_1(x)+\cdots + \beta_{K+3}b_{K+3}(x)+\epsilon, 
\] \textbf{knot placement}: General principle is that placing more knots
in places where the function might vary most rapidly. In practice, it is
common to place them at uniform quantiles of the observed \(X\). This
can be done by specifying a dof, and then let the algorithm to calculate
the knots at uniform quantiles. Note a natural spline have more internal
knots than a regression spline for the same dof.

For a \textbf{regular cubic spline}, the \textbf{basis functions} are
defined by \begin{align}
b_1(x) & = x \\
b_2(x) & = x^2 \\
b_3(x) & = x^3 \\
b_{k+3}(x) & = (x- \xi_k)^3_{+}, \qquad k=1, \cdots K
\end{align}

\textbf{dof}: \(K+4\) number of parameters (including the intercept). A
regression spline can have high variance at the outer range of the
predictors. To remedy this, one can use a \emph{natural cubic spline}.

a \textbf{natural cubic spline} extrapolates linearly ( as a linear
function) beyond the internal knots. This adds \(2\times 2\) extra
constrains. A natural cubic spline allows to put more internal knots for
the same degree of freedom as a regular cubic spline.

\textbf{dof}: \(K+2\) (\(K\) only counts the internal knots; including
the intercept).

For a \textbf{smoothing spline}: it is the solution \(g\) to the
following problem: \[
  \text{minimize}_{g\in S}\sum_{i=1}^n(y_i-g(x_i))^2 +\lambda \int g''(t)dt
  \]

The first term is the \emph{loss} RSS and encourages \(g(x_i)\) matches
\(y_i\). The second term is the \emph{penalty} that penalize the
\emph{variability} in \(g\) (measured by \(g''(t)\)) by a tuning
parameter \(\lambda \ge 0\). If \(\lambda=0\) (no constraints on \(g\)),
then the solution is just an interpolating polynomial. If
\(\lambda\to \infty\), then \(g\) is a linear function (because its
second derivative is zero). \(\lambda\) controls the bias-variance
trade-off.

The smoothing spline is in fact a natural spline with knots at unique
values of \(x_i\). But it is \textbf{different} than the natural spline.
It is a \emph{shrunk} version of a natural cubic spline, otherwise it
would have too large (\emph{nominal}) dof (number of parameters) because
it has knots at unique values of \(x_i\). It avoids the knot selection
issue and leaving a single \(\lambda\) to tune. An \emph{effective
degrees of freedom} can be calculated for a smoothing spline as \[
df_\lambda =\sum_{i=1}^n {\{\bf{S}_\lambda}\}_{ii}, 
\] where \(\bf{S}_\lambda\) is a \(n\times n\) matrix determined by
\(\lambda\) and \(x_i\) such that the vector of \(n\) fitted values can
be written as \[
\hat{\bf{g}}_\lambda =\bf{S}_\lambda \bf{y}.
\] \(df_\lambda\) decreases from \(n\) to 2 as \(\lambda\) increases
from 0 to \(\infty\).

The LOO cross-validation error can be efficiently computed by \[
\text{RSS}_{cv}(\lambda) =\sum_{i=1}^n (y_i-\hat{g}_\lambda ^{(-i)}(x_i))^2=\sum_{i=1}^n \left[ \frac{y_i-\hat{g}_\lambda (x_i)}{1-\{ {\bf S}_\lambda\}_{ii}} \right]^2
\] - Local Regression: a non-parametric method. It is similar to spline,
but allowing regions overlap. With a sliding weight function of
\emph{span} \(s\), fit separate (constant, linear, quadratic, for
instance) fits over the range of \(X\) by weighted least squares.

The span \(s\) plays the similar role as \(\lambda\) in a smoothing
spline, it controls the flexibility of the local regression. The smaller
\(s\) is, the more \emph{local and wiggle} will be the fit.

Local regression is a \emph{memory based} procedure, because like KNN,
all training data are needed each time when making a prediction.

Local regression can be generalized to \emph{varying coefficient models}
that fits a multiple linear regression model that is global in some
variables but local in another, such as time.

Local regression can be naturally extends to \(p\)-dimension using a
\(p\)-dimensional neighborhood, but really used when \(p\) is larger
than 3 or 4 because there will be generally very few training examples
near \(x_0\) (curse of dimensionality)

\begin{itemize}
\tightlist
\item
  GAM (Generalized Additive Models): can be considered as an extension
  of multiple linear regression, replacing each feature \(\beta_jX_j\)
  with an nonlinear function \(f_j(X_j)\). \[
  y_i = \beta_0 + f_1(x_{i1}) + f_2(x_{i2}) +\cdots + f_p(x_{ip})  + \epsilon
  \] GAM can mix different \(f_j\), for example, a spline, or a linear
  term or even include low order interactive terms. The coefficients are
  hard to interpret, but the fitted values are of interest.
\end{itemize}

GAM can be used in fitting a logistic regression model, that is \[
\log \frac{p(X)}{1-p(X)} =\beta_0+f_1(X_1)+f_2(X_2)+\cdots +f_p(X_p)
\]

When fitting a GAM, and if OLS can not be used (such as when a smoothing
spline is used), then the \emph{back fitting} iterative method can be
used: randomly initialize all variable coefficients; repeatedly hold all
but one variable fixed, and perform a simple linear regression on that
single variable, and update the corresponding coefficients until
convergence. Convergence is typically very fast.

\textbf{Pros and Cons of GAM} * flexible to model \(f_j\), eliminating
the need to try different transformations on each variable * potentially
more accurate prediction * because the model is additive, can easily
examine the effects of \(X_j\) on \(Y\) by holding all of the other
variables fixed. * The smoothness of \(f_j\) can be summarized by the
effective dof. * interaction terms \(X_jX_k\) can be added. * low
dimensional interaction functions of the form \(f_jk}(X_j, X_k)\) can be
added. Such term can be fit using a two-dimensional smoothers such as
local regression or two dimensional splines.

\section{Homework:}\label{homework-3}

\begin{itemize}
\tightlist
\item
  Conceptual: 1--5
\item
  Applied: At least one.
\end{itemize}

\section{Code Snippet}\label{code-snippet-1}

\subsection{Python}\label{python-5}

\begin{verbatim}

\end{verbatim}

\subsection{Numpy}\label{numpy-5}

\begin{verbatim}
Wage['education'].cat.categories # .cat is the categorical method accessor
Wage['education'].cat.codes
pd.crosstab(Wage['high_earn'], Wage['education'])

np.column_stack([Wage_['age'],
                         Wage_['year'],
                         Wage_['education'].cat.codes-1])

Xs = [ns_age.transform(age),
      ns_year.transform(Wage['year']),
      pd.get_dummies(Wage['education']).values] # -> 5 education levels: 1-hot coding
X_bh = np.hstack(Xs)
\end{verbatim}

\subsection{Pandas}\label{pandas-5}

\begin{verbatim}
cut_age = pd.qcut(age, 4) # cut based on the 25%, 50%, and 75% cutpoints. pd.cut is similar
\end{verbatim}

\subsection{Graphics}\label{graphics-5}

\begin{verbatim}
ax.legend(title='$\lambda$');
\end{verbatim}

\subsection{ISLP and statsmodels}\label{islp-and-statsmodels-3}

\begin{verbatim}

\end{verbatim}

\subsection{sklearn}\label{sklearn-3}

\subsection{Useful code snippets}\label{useful-code-snippets}

\subsubsection{plot a model fit with confidence
interval}\label{plot-a-model-fit-with-confidence-interval}

\begin{verbatim}
def plot_wage_fit(age_df, 
                  basis, # ISL model object
                  title):

    X = basis.transform(Wage)
    Xnew = basis.transform(age_df)
    M = sm.OLS(y, X).fit()
    preds = M.get_prediction(Xnew)
    bands = preds.conf_int(alpha=0.05)
    fig, ax = subplots(figsize=(8,8))
    ax.scatter(age,
               y,
               facecolor='gray',
               alpha=0.5)
    for val, ls in zip([preds.predicted_mean,
                      bands[:,0],
                      bands[:,1]],
                     ['b','r--','r--']):
        ax.plot(age_df.values, val, ls, linewidth=3)
    ax.set_title(title, fontsize=20)
    ax.set_xlabel('Age', fontsize=20)
    ax.set_ylabel('Wage', fontsize=20);
    return ax
\end{verbatim}

\subsubsection{Fitting with a step
function}\label{fitting-with-a-step-function}

\begin{verbatim}
cut_age = pd.qcut(age, 4) # cut based on the 25%, 50%, and 75% cutpoints
# note pd.get_dummies(cut_age) is the X matrix
summarize(sm.OLS(y, pd.get_dummies(cut_age)).fit()) 
\end{verbatim}

\subsubsection{Fitting a spline}\label{fitting-a-spline}

\begin{verbatim}
#specifying internal knots

bs_age = MS([bs('age',
                internal_knots=[25,40,60],
                name='bs(age)')]) #rename the variable names 
Xbs = bs_age.fit_transform(Wage) # Xbs == bs_age above
M = sm.OLS(y, Xbs).fit()
summarize(M)

# specifying df 
bs_age0 = MS([bs('age',
                 df=3, # df count does not include intercept. df=degree+ #knots
                 degree=0)]).fit(Wage)
Xbs0 = bs_age0.transform(Wage)
summarize(sm.OLS(y, Xbs0).fit())

BSpline(df=3, degree=0).fit(age).internal_knots_

# Fit a natural spline
ns_age = MS([ns('age', df=5)]).fit(Wage) #df=degree+ #knots -2
M_ns = sm.OLS(y, ns_age.transform(Wage)).fit()
summarize(M_ns)

# fit a smoothing spline
X_age = np.asarray(age).reshape((-1,1))
gam = LinearGAM(s_gam(0, lam=0.6)) #gam is the smoothing spline model with a given lambda
gam.fit(X_age, y)

#Fiting a smoothing spline with an optimized lambda
gam_opt = gam.gridsearch(X_age, y)


# Fitting a smoothin spline by specifying a df (not including intercept)
fig, ax = subplots(figsize=(8,8))
ax.scatter(X_age,
           y,
           facecolor='gray',
           alpha=0.3)
for df in [1,3,4,8,15]:
    lam = approx_lam(X_age, age_term, df+1) # find the lambda corresponding to a df. 
    age_term.lam = lam # update lambda
    gam.fit(X_age, y)
    ax.plot(age_grid,
            gam.predict(age_grid),
            label='{:d}'.format(df),
            linewidth=4)
ax.set_xlabel('Age', fontsize=20)
ax.set_ylabel('Wage', fontsize=20);
ax.legend(title='Degrees of freedom');



\end{verbatim}

\subsection{GAM}\label{gam}

\begin{verbatim}
### manually contruct basis 
ns_age = NaturalSpline(df=4).fit(age) #df counts do not include intercepts. -> 4 columns
ns_year = NaturalSpline(df=5).fit(Wage['year']) # -> 5 cols
Xs = [ns_age.transform(age),
      ns_year.transform(Wage['year']),
      pd.get_dummies(Wage['education']).values] # -> 5 education levels: 1-hot coding
X_bh = np.hstack(Xs)
gam_bh = sm.OLS(y, X_bh).fit()

### Examinge partial effect

age_grid = np.linspace(age.min(),
                       age.max(),
                       100)
X_age_bh = X_bh.copy()[:100] # Take the first 100 rows of X_bh
# calculate the row mean and make it a row vector in the shape of 1Xp, then broadcast
X_age_bh[:] = X_bh[:].mean(0)[None,:] 
X_age_bh[:,:4] = ns_age.transform(age_grid)# replace the first 4 cols with basis functions evalued at age_grid
preds = gam_bh.get_prediction(X_age_bh) #gam_bh is the GAM model with all 14 basis
bounds_age = preds.conf_int(alpha=0.05)
partial_age = preds.predicted_mean
center = partial_age.mean() # center of the prediction 
partial_age -= center # center the prediction for better viz
bounds_age -= center
fig, ax = subplots(figsize=(8,8))
ax.plot(age_grid, partial_age, 'b', linewidth=3)
ax.plot(age_grid, bounds_age[:,0], 'r--', linewidth=3)
ax.plot(age_grid, bounds_age[:,1], 'r--', linewidth=3)
ax.set_xlabel('Age')
ax.set_ylabel('Effect on wage')
ax.set_title('Partial dependence of age on wage', fontsize=20);


### Using a smoothing spline and pygam package
#### Specifying lambda
#### default \lambda = 0.6 is used.
gam_full = LinearGAM(s_gam(0) + # spline smoothing applies to the first col of the feature matrix
                     s_gam(1, n_splines=7) + # smoothing applied to the 2nd col 
                     f_gam(2, lam=0)) # smothing applied to the 3rd col: a factor col
Xgam = np.column_stack([age,  #stack as columns
                        Wage['year'],
                        Wage['education'].cat.codes]) 
gam_full = gam_full.fit(Xgam, y)

gam_full.summary() # verbose summary

#### Plot partial effect using a plot_gam from ISLP.pygam
fig, ax = subplots(figsize=(8,8))
plot_gam(gam_full, 0, ax=ax) # 0: partial plot of the first component: age
ax.set_xlabel('Age')
ax.set_ylabel('Effect on wage')
ax.set_title('Partial dependence of age on wage - default lam=0.6', fontsize=20);

### Specifying df
age_term = gam_full.terms[0]
age_term.lam = approx_lam(Xgam, age_term, df=4+1)
year_term = gam_full.terms[1]
year_term.lam = approx_lam(Xgam, year_term, df=4+1)
gam_full = gam_full.fit(Xgam, y)

#### Plot partial effect
fig, ax = subplots(figsize=(8, 8))
ax = plot_gam(gam_full, 2)
ax.set_xlabel('Education')
ax.set_ylabel('Effect on wage')
ax.set_title('Partial dependence of wage on education',
             fontsize=20);
ax.set_xticklabels(Wage['education'].cat.categories, fontsize=8);
\end{verbatim}

\subsubsection{Anova for GAM}\label{anova-for-gam}

\begin{verbatim}
gam_0 = LinearGAM(age_term + f_gam(2, lam=0)) # note age_term is a s_gam with df=4 defined above 
gam_0.fit(Xgam, y)
gam_linear = LinearGAM(age_term +
                       l_gam(1, lam=0) +
                       f_gam(2, lam=0))
gam_linear.fit(Xgam, y)
anova_gam(gam_0, gam_linear, gam_full)
\end{verbatim}

\subsubsection{Logistic GAM}\label{logistic-gam}

\begin{verbatim}
gam_logit = LogisticGAM(age_term + 
                        l_gam(1, lam=0) +
                        f_gam(2, lam=0))
gam_logit.fit(Xgam, high_earn)
\end{verbatim}

\subsubsection{LOESS}\label{loess}

\begin{verbatim}
lowess = sm.nonparametric.lowess
fig, ax = subplots(figsize=(8,8))
ax.scatter(age, y, facecolor='gray', alpha=0.5)
for span in [0.2, 0.5]:
    fitted = lowess(y,
                    age,
                    frac=span,
                    xvals=age_grid)
    ax.plot(age_grid,
            fitted,
            label='{:.1f}'.format(span),
            linewidth=4)
ax.set_xlabel('Age', fontsize=20)
ax.set_ylabel('Wage', fontsize=20);
ax.legend(title='span', fontsize=15);
\end{verbatim}

\bookmarksetup{startatroot}

\chapter{Class Project}\label{class-project}

\textbf{Goal}

to use various ML algorithms to predict a meaningful target \(Y\) by
\emph{classification algorithms} or \emph{regression algorithms}.
Present it at COS Research Sympsium in the end of April.

\textbf{Data set}: real-world stock price and volume data. We could
start with just one stock, e.g.~APAL, S\&P 500, index, DJ index.

\textbf{Some ideas}: * Create one model for each stock. * create a
single model for all stocks. Needs to embed each stock in a feature
space. Research?

\textbf{Start-up code}: refer to the page
\url{https://github.com/ywanglab/Predicting_stock_movement/blob/main/Time_series_stock_data_analysis_ver2.ipynb}.
Perform some EDA to feel the data.

reference ticker symbols:
\url{https://gist.github.com/quantra-go-algo/ac5180bf164a7894f70969fa563627b2}

\textbf{Questions}: Which are the \(X\) variables? price, volume,
return, day of week, month of year, etc. What is the \(Y\) variable?
next-day price, next-day return, next-five-day average price,
next-five-day-average return, etc.

\textbf{Models}

\begin{itemize}
\item
  Linear regression:

  \begin{itemize}
  \tightlist
  \item
    including continuous variables (price, volume), categorical
    variables (day-of-week, month-of-year)
  \item
    transforming \(X\) (for including non-linear relation between \(Y\)
    and \(X\)) or \(Y\) (when \(Y\) is heteroschedatic, i.e., with
    varied \(\epsilon_i\))
  \item
    plot residual plot to see if \(Var(\epsilon_i)\) is changing. If
    yes, may appeal to transforming \(Y\), e.g., \(\log Y\),
    \(\sqrt{Y}\).
  \item
    investigate outliers (points with unusual \(Y\)-values) using the
    \emph{residual plot} or looking at \emph{studentized residual}.
  \item
    investigate high leverage points (with unusual \(x\) values), by
    calculating \emph{leverage statistics}.
  \item
    Investigate if there is \emph{colinearity} among the variables by
    calculating \emph{VIF}.
  \end{itemize}
\item
  classification: predicting directions of the stock price movement.
  binary (with two direction), or multinomial (more than two values:
  e.g., up, same, down), LDA, QDA
\item
  regularization of the parameters
\item
  selection of variables: forward, backward, mixture, regularization,
  cross-validation
\item
  decision tree: random forest, boosting
\item
  SVM: support vector machines
\item
  NN
\end{itemize}

\bookmarksetup{startatroot}

\chapter{Summary}\label{summary}

In summary, this book has no content whatsoever.

\bookmarksetup{startatroot}

\chapter*{References}\label{references}
\addcontentsline{toc}{chapter}{References}

\markboth{References}{References}

James et al. (2023)

\phantomsection\label{refs}
\begin{CSLReferences}{1}{0}
\bibitem[\citeproctext]{ref-thrt23}
James, G., D. Witten, T. Hastie, R. Tibshirani, and J. Taylor. 2023.
\emph{An Introduction to Statistical Learning}. USA: Springer.
\url{https://hastie.su.domains/ISLP/ISLP_website.pdf}.

\end{CSLReferences}



\end{document}
